\documentclass[english,course]{lecturenotes}
\usepackage{lipsum}
\usepackage{enumitem}
\usepackage{amsmath}
\usepackage{framed}
\usepackage{tikz}
\usetikzlibrary{calc}
\usepackage{amsmath,amssymb}
\usepackage{tikz-cd}
\usetikzlibrary{fit}
\usepackage{mdframed}
\usepackage{float}
\usetikzlibrary{arrows.meta}
\usepackage{enumitem}
\usetikzlibrary{arrows.meta,decorations.pathreplacing,calc}
\usepackage{xcolor}
\usetikzlibrary{decorations.markings}
\usepackage{pgfplots}
\pgfplotsset{compat=1.18}
\usepackage{graphicx}
\usepackage{caption}
\usepackage{subcaption}


\usetikzlibrary{external}
\tikzexternalize[prefix=cache/]


\tikzset{external/system call={
		xelatex \tikzexternalcheckshellescape -halt-on-error -interaction=batchmode
		-jobname "\image" "\texsource"}}

\title{Algebraic Curves}
\newtheorem*{example*}{Example}
\newtheorem*{proposition*}{Proposition}
\newtheorem*{definition*}{Definition}
\newtheorem*{lemma*}{Lemma}
\newtheorem*{theorem*}{Theorem}
\newtheorem*{question}{Question}
\newtheorem*{scratch}{Scratch of Proof}

\shorttitle{Shortened title} % For headers; if undefined, the usual title will be used
% Most of these data are not compulsory
\subject{Subject of the Talk}
\author{TSE-YU SU}


\speaker{}
\date{21}{11}{2025}


\begin{document}
	\section{AFFINE ALGEBRAIC SETS}
	\subsection{Affine Space and Algebraic Sets}
	\begin{definition}
		Let $k$ be a field, we define the \textbf{affine n-space} $\mathbb{A}^n(k)$,or  $\mathbb{A}^n$ if $k$ is clear, to be the set of n-tuples of elements of $k$.
	\end{definition}	
	If $F\in k[X_1,...,X_n]$, a point $P=(a_1,...,a_n)$ is said to be a \textbf{zero} of $F$ if\\ $F(P)=F(a_1,...,a_n)=0$. For $F\neq const$, the set of zeros of $F$ is called the \textbf{hypersurface} defined by $F$, and is denoted by $V(F)$. If $F$ is of degree 1, $V(F)$ is called a \textbf{hyperplane} in $\mathbb{A}^n$.\\
	For $S$: a set of polynomials in $k[X_1,...,X_n]$, define $V(S)$ to be the set of all common zeros of $F\in S$.\\
	A subset $X\subseteq \mathbb{A}^n(k)$ is called an \textbf{affine algebraic set}, or simply \textbf{algebraic set}, if $X=V(S)$ for some $S$.\\
	Can verify that the set of all algebraice sets in $\mathbb{A}^n$ forms a topology, called \textbf{Zariski topology}, where the closed sets are exactly the algebraic sets.\\
	Here are some facts about algebraic sets:
	\begin{itemize}
		\item If $I$ is the ideal in $k[X_1,...,X_n]$ generated by $S$, then $V(S)=V(I)$. Therefore we can restrict to the case $V(I)$.
		\item If $\lbrace I_\alpha \rbrace$ is \textbf{any} collection of ideals, then $V(\cup_\alpha I_\alpha)=\cap_\alpha V(I_\alpha)$. So the intersection of \textbf{any} collection of algebraic sets is also an algebraic set.
		\item If $I\subseteq J$, then $V(I)\supseteq V(J)$.
		\item For polynomials $F,G$, $V(F)\cup V(G)=V(FG)$, and for ideals $I,J$, $V(I)\cup V(J)=V(IJ)$. So any \textbf{finite union} of algebraic sets is also algebraic.
		\item $V(0)=\mathbb{A}^n,V(1)=V(k[X_1,...,X_n])=\emptyset$. And $V(X_1-a_1,...,X_n-a_n)=\lbrace  (a_1,...,a_n)\rbrace$. So any finite subset of $\mathbb{A}^n$ is algebraic.
		\item $V(IJ)=V(I\cap J)$. The $\supseteq$ is clear, now prove $\subseteq$. Let $P\in V(IJ)$, then for every $F\in I,G\in J$, $FG(P)=0$. Hence for $H\in I\cap J$, $H^2(P)=0$, so $H(P)=0$. 
	\end{itemize}
	\subsection{The Ideal of a Set of Points}
	For any $X\subseteq \mathbb{A}^n$, consider those polynomials in $k[X_1,...,X_n]$ which vanish on $X$, these polynomials form an ideal $\triangleleft k[X_1,...,X_n]$, called the ideal of $X$, written $I(X)=\lbrace F\in k[X_1,...,X_n] | F(X)=\lbrace 0\rbrace  \rbrace$.
	\\The following properties show some relation between ideals and algebraic sets.
	\begin{itemize}
		\item If $X\subseteq Y$, then $I(X) \supseteq I(Y)$
		\item $I(\emptyset)=k[X_1,...,X_n]$. $I(\mathbb{A}^n)=0$. $I(\mathbb{A}^n)=0$ if $k$ is not finite. A counterexample is $0\neq x(x-1)\in I(\mathbb{A}^1(\mathbb{Z}_2))\triangleleft \mathbb{Z}_2[x]$
		\item $I(\lbrace a_1,...,a_n \rbrace)=(X_1-a_1,...,X_n-a_n)$
		\item $S\subseteq I(V(S))$ and $X\subseteq V(I(X))$
		\item $V(I(V(S)))=V(S)$ and $I(V(I(X)))=I(X)$.
	\end{itemize}
	An ideal which is the ideal of an algebraic set, satisfies the following property:
	\begin{equation*}
		\text{If }I=I(X), \text{ and }F^n\in I \text{ for some }n\in\mathbb{N}, \text{ then }F\in I.
	\end{equation*}
	Consequently, $I(X)=\text{Rad}(I(X))=\sqrt{I(X)}$ is radical. 
	\subsection{The Hilbert Basis Theorem}
	We defined an algebraic set by any set of polynomials, but in fact finitely many will suffice.
	\begin{theorem}
		Every algebraic set is the intersection of a finite number of hypersurfaces.
	\end{theorem}
	In order to prove this theorem, it suffices to show any ideal $I\triangleleft k[X_1,...,X_n]$ is finitely generated by $(F_1,...,F_s)$, then $V(I)=V(F_1)\cap V(F_2)\cap ... \cap V(F_n)$.
	\begin{theorem}[Hilbert Basis Theorem]
		If $R$ is a Noetherian ring, then $R[X]$ is also Noetherian. Consequetly, $R[X_1,...,X_n]$ is Noetherian.
	\end{theorem}
	\subsection{Irreducible Components of an Algebraic Set}
	An algebraic set may be the union of serveral smaller algebraic sets. An algebraic set $V\subseteq \mathbb{A}^n$ is said to be \textbf{reducible} if $V=V_1\cup V_2$, where $V_1\neq V \neq V_2$ are algebraic. Otherwise $V$ is \textbf{irreducible}.
	\begin{proposition}
		An algebraic set $V$ is irreducible if and only if $I(V)$ is prime.
	\end{proposition}
	We want to show that an algebraic set is the union of finitely many irreducible algebraic sets. If $V$ is reducible, write $V=V_1\cup V_2$, if $V_2$ is reducible, write $V_2=V_3\cup V_4$, need to show this process stops.\\
	Since $k[X_1,...,X_n]$ is Noetherian, each set of ideals has an maximal element, consequently, any collection of algebraic sets in $\mathbb{A}^n$ has an minimal element.
	\begin{theorem}
		Let $V$ be an algebraic set in $\mathbb{A}^n(k)$, then there are unique irreducible algebraic sets $V_1,...,V_m$ such that $V=V_1\cup V_2 \cup ... \cup V_m$ and $V_i\nsubseteq V_j$ for $i\neq j$.
	\end{theorem}
	\begin{proof}
	Let $\mathcal{S}$ be the set of all algebraic sets $V\subseteq \mathbb{A}^n$ which is not the union of a finite number of irreducible. Choose an minimal element $V$ in $\mathcal{S}$, clearly $V$ is reducible, say $V=V_1\sup V_2$, where $V_i\neq V$. But then $V_i\subsetneq V$, so $V_i$ is a union of finitely many irreducible algebraic sets, hence so is $V=V_1\cup V_2$, a contradiction. To show $V_i\nsubseteq V_j$ for $i\neq j$, simply delete every algebraic set which is contained in another bigger algebraic set.\\
	To show uniqueness, let $V=W_1\cup ... \cup W_s$. Since $V_i=V\cap V_i=\cup_j (W_j\cap V_i)$, and $V_i$ is irreducible, $V_i\subseteq W_{j(i)}$ for some $j(i)$. Similarly, $W_{j(i)}\subseteq V_{k(j(i))}$ for some $k(j(i))$, but then $V_i\subseteq V_{k(j(i))}$, hence $i=k(j(i))$ and $V_i=W_{j(i)}$. Likewise $W_j=V_{i(j)}$ for some $i(j)$, so $s=n$ and $W_i=V_i$ after renumbering.
	\end{proof}
	These $V_1,...,V_n$ are called the irreducible components of $V$, and $V=V_1\cup ... \cup V_n$ is the decomposition of $V$ into irreducible components.
	\subsection{Algebraic Subsets of the Plane}
	Will classify all irreducible algebraic sets of $\mathbb{A}^2(k)$ in this subsection. Once this classification has been done, by Theorem 1.2 we have found all algebraic sets.\\
	\begin{proposition}
		Let $F,G\in k[X,Y]$ with no common factors. Then $V(F,G)=V(F)\cap V(G)$ is a finite set of points.
	\end{proposition}
	\begin{proof}
		Consider $A=(F,G)\cap k[X]$, $A$ is an ideal of $k[X]$. Since $k[X]$ is PID, $A=(f(X))$. So $FH+GK=f(X)$ for some $H,K\in k[X,Y]$. Thus the $X-$component of points in $V(F,G)$ are roots of $f(X)$, which is finitely many. Similarly, the $Y-$component of points in $V(F,G)$ are roots of some $g\in k[Y]$. Hence $V(F,G)\subseteq \lbrace (a,b) | f(a)=g(b)=0 \rbrace$, which is finite.
	\end{proof}
	\begin{corollary}
		If $F$ is irreducible in $k[X,Y]$, and if $V(F)$ is infinite, then $I(V(F))=(F)$ and $V(F)$ is irreducible.
	\end{corollary}
	\begin{proof}
		Take $G\in I(V(F))$, clearly $G(V(F))=\lbrace 0 \rbrace$, hence $V(F)\subseteq V(F,G)$, and $V(F,G)$ is infinite. By the previous proposition $F,G$ must have common factor, since $F$ is irreducible this common factor can only be $F$, so $G\in (F)$, and thus $I(V(F))=(F)$, and by proposition 1.4 $V(F)$ is irreducible.
	\end{proof}
	\begin{corollary}
		Suppose $k$ is infinite, then the irreducible algebraic subsets of $\mathbb{A}^2(k)$ are:
		\begin{align*}
			&\mathbb{A}^2(k),\\
			&\emptyset,\\
			&\textbf{points},\\
			&\textbf{irreducible plane curves }V(F)
		\end{align*}
	where $F$ is an irreducible polynomial and $V(F)$ is infinite.
	\end{corollary}
	\textbf{Note:} Not all zero sets of irreducible polynomial in $k[X,Y]$ is infinte, for example $X^2+Y^2\in\mathbb{R}[X,Y]$ is irreducible, but the zero set $\lbrace (0,0) \rbrace$ is finite.
	\begin{corollary}
		Assume $k$ is algebraically closed, and $F\in k[X,Y]$. Let $F=F_{1}^{n_1}...F_{r}^{n_r}$ be the decomposition of $F$ into irreducible factors. Then $V(F)=V(F_1)\cup ... \cup V(F_r)$ is the decomposition of $F$ into irreducible components, and $I(V(F))=(F_1F_2...F_r)$.
	\end{corollary}
	\begin{proof}
		$V(F)=V(F_1)\cup ... \cup V(F_r)$ is clear. Since $k$ is algebraically closed, $V(F_i)$ is infinite, and by the previous corollary $V(F_i)$ is irreducible.\\
		(Note: The cases such as $X^2+Y^2\in \mathbb{R}[X,Y]$, which is irreducible but has finite zero set, won't happen.)\\ 
		Also, since $F_i \nmid F_j$, there's no inclusion relation among $V(F_i)$.\\
		The next part $I(V(F))=(F_1F_2...F_r)$ is also clear.
		
	\end{proof}
	The following problem shows why we usually require $k$ to be algebraically closes.
	\begin{question}
		Show that every algebraic subset of $\mathbb{A}^2(\mathbb{R})$ is equal to some $V(F)$, where $F\in\mathbb{R}[X,Y]$.
	\end{question}
	\begin{proof}
		It suufices to show any finite set of points $\lbrace (a_1,b_1),...,(a_r,b_r) \rbrace$ in $\mathbb{A}^2(\mathbb{R})$ can be written as $V(F)$ for some $F\in\mathbb{R}[X,Y]$.\\
		Since $(X-a)^2+(Y-b)^2$ has only one zero $(a,b)$ in $\mathbb{A}^2(\mathbb{R})$, $F=\prod_{i=i}^{r}((X-a_i)^2+(Y-b_i)^2)$ is the desired polynomial.
	\end{proof}
	\subsection{Hilbert's Nullstellensatz}
		\begin{center}\fbox{we assume $k$ is algebraically closed in this subsection}.\end{center}\\
		Want to find the exact relation between algebraic sets and ideals. Will first prove a weaker theorem: 
		
	\begin{theorem}[Weak Nullstellensatz]
		If $I$ is a proper ideal in $k[X_1,...,X_n]$, then $V(I)\neq \emptyset$.
	\end{theorem}
	\begin{proof}
		Since $I$ is contained in some maximal ideal $\mathfrak{m}$, and $V(\mathfrak{m})\subseteq V(I)$, it suffices to show for every maximal ideals $\mathfrak{m}$, $V(\mathfrak{m})\neq \emptyset$.\\
		Will use the following fact:\\
		\textbf{Fact:} If $k$ is algebraically closed, then maximal ideals of $k[X_1,...,X_n]$ are of the form $(X_1-a_1,...,X_n-a_n)$.\\
		By the above fact $V(X_1-a_1,...,X_n-a_n)=\lbrace (a_1,...,a_n) \rbrace\neq \emptyset$.
	\end{proof}
	\begin{theorem}[Hilbert's Nullstellensatz]
		Let $I$ be an ideal in $k[X_1,...,X_n]$, $k$ is algebraically closed. Then $I(V(I))=\text{Rad}(I)$.
	\end{theorem}
	\begin{proof}
		Rad$(I)\subseteq I(V(I))$ is easy. For another direction, suppose $G\in I(V(F_1,...,F_r)),$ $F_i\in k[X_1,...,X_n]$, let $J=(F_1,...,F_r,X_{n+1}G-1)\subseteq k[X_1,...,X_n,X_{n+1}]$, can see $\emptyset=V(J)\subseteq \mathbb{A}^{n+1}.$ Apply Weak Nullstellensatz to $J$, $J=k[X_1,...,X_n,X_{n+1}]$. So $1=\sum A_i(X_1,...,X_{n+1})F_i+B(X_1,...,X_{n+1})\cdot(X_{n+1}G-1)$.\\
		Let $Y=\frac{1}{X_{n+1}}$, multiply the above equation sufficiently many times by $Y$, that the $X_{n+1}$-degree of each monomial terms is negative. (For example, $X_1 X_{n+1}^{3}+X_{2}^{3} X_{n+1}^{5}\xrightarrow{\times Y^5}X_1Y^2+X_{2}^{3}=P(\lbrace X_i|i=1,...,n \rbrace,Y)$ )\\
		Then we get an equation $Y^N=\sum C_i(X_1,...,Y)F_i+D(X_1,...,X_n,Y)\cdot(G-Y)\in k[X_1,...,X_n,Y]$, substitute $Y=G$, it follows that $G^N\in (F_1,...,F_r)$.
	\end{proof}
	Here are some immediate corollary, for $k:$ algebraically closed:
	\begin{corollary}
		There is a one-to-one correspondence between \textbf{radical ideals} and \textbf{algebraic sets}.
	\end{corollary}
	\begin{corollary}
		If $I$ is prime, then $V(I)$ is irreducible. There is a one-to-one correspondence between \textbf{prime ideals} and \textbf{irreducible algebraic sets}. The maximal ideals correspond to points.
	\end{corollary}
	\begin{corollary}
		Let $F=F_{1}^{n_1}...F_{r}^{n_r}$ be the decomposition of $F$ into irreducible factors, then $V(F)=V(F_1)\cup ... \cup V(F_r)$ is the decomposition of $V(F)$ into irreducible components, and $I(V(F))=(F_1F_2...F_r)$. \\
		There is a one-to-one correspondence between \textbf{irreducible polynomials}(up to multiplying by a unit) and \textbf{irreducible hypersurfaces} in $\mathbb{A}^n(k)$. Remenber that a hypersurface is the zero set of a polynomial.
	\end{corollary}
	\begin{align*}
		\text{Radical ideals }&\leftrightarrow \text{ Algebraic sets}\\
		\text{Prime ideals }&\leftrightarrow \text{ Irreducible algebraic sets}\\
		\text{Irreducible polynomials }&\leftrightarrow \text{ Irreducible hypersurfaces}
	\end{align*}
	\begin{corollary}
		Let $I$ be an ideal in $k[X_1,...,X_n]$, then $V(I)$ is a finite set if and only if $k[X_1,...,X_n]/I$ is a finite dimensional vector space over $k$. In this case the number of points in $V(I)$ is less or equal to $ \text{dim}_k(k[X_1,...,X_n]/I)$.
	\end{corollary}
	\begin{proof}
		Assume $k[X_1,...,X_n]/I$ is a finite dimensional vector space over $k$, let points $P_1,...,P_r\in V(I)$, choose polynomials $F_i\in k[X_1,...,X_n],i=1,...,r$ s.t. $F_i(P_i)=1$ and $F_i(P_j)=0$ for $i\neq j$. Want to show $I-$redidue classes $\bar{F_i}$ are linearly independent over $k$. If $\sum_{i=1}^{r}\lambda_i \bar{F_i}=0$, then $\sum_{i=1}^{r}\lambda_i F_i\in I$, so $\lambda_j=\sum_{i=1}^{r}\lambda_i F_i(P_j)=0$, hence $\bar{F_i}$ are linearly independent over $k$, so $r\leq \text{dim}_k(k[X_1,...,X_n]/I)$.\\
		Conversely, if $V(I)=\lbrace P_1,...,P_r \rbrace$ is finite, let $P_{i}=(a_{i,1},...a_{i,n})$, and for $j=1,...,n$ define $F_j=\prod_{s=1}^{r}(X_j-a_{s,j})$, clearly $F_j\in V(I)$, so by Nullstellensatz $F_j^N\in I$ for some $N$, WLOG take $N$ so large that it holds for all $F_j$. Consequently $\bar{F_{j}^{N}}=0$, for all $j$ and
		 since $F_{j}^{N}\in I$ is a polynomial in $X_j$ of degree $rN$, $\bar{X_j}^{rN}$ is a $k-$linear combination of $1,X_{j}^{1},X_{j}^{2},...,X_{j}^{rN-1}$, and hence so is any positve order $X_{j}^{s}$. Therefore $\lbrace \bar{X_1}^{m_1}\cdot \bar{X_2}^{m_2}\cdot...\cdot \bar{X_n}^{m_n} \rbrace$ generates $k[X_1,...,X_n]/I$ as a vector space over $k$.
	\end{proof}
	
	
	\section{AFFINE VARIETIES}
	\subsection{Coordinate Rings}
	Recall that if $V\subseteq \mathbb{A}^n$ is a variety, then $I(V)$ is a prime ideal and $k[X_1,...,X_n]/I(V)$ is a domain. We denote $\Gamma (V)=k[X_1,...,X_n]/I(V)$ and call it the \textbf{coordinate ring} of $V$.\\
	For any $S\neq \emptyset$, let $\mathcal{F}(V,k)$ be the ring of all functions from $V$ to $k$, with the obvious addition and multiplication $f\cdot g(x)=f(x)\cdot g(x)$.\\
	If $V\in \mathbb{A}^n$ is a variety, $f\in\mathcal{F}(V,k)$ is called a \textbf{polynomial function} if there is a polynomial $F\in k[X_1,...,X_n]$ s.t. $F(a_1,...,a_n)=f(a_1,...,a_n)$ for all $(a_1,...,a_n)\in V$. These polynomial functions form a subring of $\mathcal{F}(V,k)$ containing $k$ (the identity functions), and two polynomial $F,G$ determine the same function iff $F-G$ vanishes on $V$, i.e. $F-G\in I(V)$. Thus we consider $\Gamma(V)$ as a subring of $\mathcal{F}(V,k)$ consisting of all polynomial functions on $V$.
	\subsection{Polynomial Maps}
	Let $V\subseteq \mathbb{A}^n\,,W\subseteq\mathbb{A}^m$, a mapping $\varphi: V\rightarrow W$ is called a \textbf{polynomial map} if there are polynomials $T_1,...,T_m\in k[X_1,...,X_n]$ s.t. $\varphi(a_1,...,a_n)=(T_1(a_1,...,a_n),...,T_m(a_1,...,a_n))$ for all $(a_1,...,a_n)$.\\
	Any mapping $\varphi:V\rightarrow W$ induces a homomorphism $\tilde{\varphi}:\mathcal{F}(W,k)\rightarrow \mathcal{F}(V,k)$, by letting $\tilde{\varphi}(f)=f\circ \varphi$.
	\begin{center}
		\begin{tikzpicture}
			\node (W) at (0,0) {$W$};
			\node (V) at (0,-1) {$V$};
			\node (k) at (1.5,0) {$k$};
			
			\draw[->] (W)--(k) node[midway,above]{$f$};
			\draw[->] (V)--(W) node[midway,left]{$\varphi$};
			\draw[->,dashed] (V)--(k) node[midway,right]{$f\circ\varphi$};
		\end{tikzpicture}
	\end{center}
	If $\varphi$ is a polynomial map, then $\tilde{\varphi}(\Gamma(W))\subseteq \Gamma(V)$, i.e. $\tilde{\varphi}$ sends polynomial functions on $W$ to polynomial functions on $V$. To show this, need only to check that $\tilde{\varphi}$ is well defined on residue class $f+I(W)\in \Gamma(W)$. Let $g\in I(W)$, $\tilde{\varphi}(g)=g\circ \varphi$, since $\varphi$ maps $V$ to $W$, $g\circ\varphi$ vanished on $V$, and the result follows.\\
	If $V=\mathbb{A}^n,\, W=\mathbb{A}^m$, and $T_1,...,T_m$ determine a polynomial map $T:\mathbb{A}^n\rightarrow\mathbb{A}^m$, then $T_i$ are unique determined by $T$, i.e. there are no other distinct $T^{\prime}_{1},...,T^{\prime}_{m}$ whcih induced the same polynomial map from $\mathbb{A}^n$ to $\mathbb{A}^m$.
	\begin{proposition}
		Let $V\subseteq \mathbb{A}^n, W\subseteq \mathbb{A}^m$ be affine varieties. There is a natural one-to-one correspondence between the polynomial maps $\varphi:V\rightarrow W$ and the homomorphisms $\tilde{\varphi}:]=\Gamma(W)\rightarrow\Gamma(V)$. Any such $\varphi$ is the restriction of a polynomial map from $\mathbb{A}^n$ to $\mathbb{A}^m$.
	\end{proposition} 
	\begin{proof}
		Let $\alpha:\Gamma(W)\rightarrow\Gamma(V)$ be a homomorphism, choose $T_1,...,T_m\in k[X_1,...,X_n]$ s.t. $T_i+I(V)=\alpha(X_i+I(W))+I(V)$ in $\Gamma(V)$, then $T=(T_1,...,T_m)$ is a polynomial map from $\mathbb{A}^n$ to $\mathbb{A}^m$, which induces $\tilde{T}:k[X_1,...,X_m]\rightarrow k[X_1,...,X_n]$. To check $T$ restrict to $V\rightarrow W$, observe that for $f(X_1,...,X_m)\in I(W)$, $\tilde{T}(f)=f\circ T=f(T_1,...,T_m)\equiv\alpha(f(X_1,...,X_m))\equiv 0$ in $\Gamma(V)$, thus $T(V)\subseteq W$. Finally, it is easy to verify $\widetilde{T|_V}=\alpha$.
	\end{proof}
	A polynomial map $\varphi: V\rightarrow W$ is an isomorphism if there is $\psi: W\rightarrow V$ s.t. $\varphi \circ \psi$ and $\psi \circ\varphi$ are identities on $W,V$ respectively. The previous proposition shows that two affine variety are isomorphic iff their coordinate rings are isomorphic over $k$.
	\begin{framed}
		\noindent\textbf{A useful test for irrducibily:}
		Let $\varphi:V\rightarrow W$ be a polynomial map, and $X\subseteq W$ is a algebraic subset, then $\varphi^{-1}(X)$ is also an algebraic subset in $V$. Moreover, if $\varphi^{-1}(X)$ is irreducible and $X$ is contained in the image $\varphi(V)$, then $\varphi^{-1}(X)$ is also irreducible.\\
		For example, $V=V(XZ-Y^2,YZ-X^3,Z^2-X^2Y)=\lbrace (t^3,t^4,t^5)|t\in k \rbrace$ is the image of $\varphi: \mathbb{A}^1\rightarrow \mathbb{A}^3$, $\varphi(t)=(t^3,t^4,t^5)$, since $\mathbb{A}$ is irreducible, so is $V$.\\
		
	\end{framed}
	\subsection{Coordinate Changes}
	If $T=(T_1,...,T_m)$ is a polynomial map from $\mathbb{A}^n$ to $\mathbb{A}^m$, and $F\in k[X_1,...,X_m]$, we denote $F^T=\tilde{T}(F)=F(T_1,...,T_m)\in k[X_1,...,X_n]$. For ideal $I$ and algebraic set $V$ in $\mathbb{A}^m$, $I^T$ denote the ideal $\triangleleft$ $k[X_1,...,X_n]$ generated by $F^T$, $F\in I$, and $V^T$ the algebraic set $V(I^T)$, where $I=I(V)$. If $V$ is the hypersurface of $F$, then $V^T$ is the hypersurface of $F^T$, for $F\neq$ constant.\\
	An \textbf{affine change} of coordinates on $\mathbb{A}^n$ is a \textbf{invetible} polynomial map $T=(T_1,...,T_n):\mathbb{A}^n\rightarrow \mathbb{A}^n$ s.t. each $T_i$ is a polynomial \textbf{of degree 1}. Not hard to see that $T$ is a composition of a linear map $\tilde{T}: X_j\rightarrow \sum_i a_{ij}X_i$  and a translation $T_a:v\mapsto v+a$. Since any translation has inverse, it follows that $\tilde{T}$ is also invertible.
	
	\subsection{Rational Functions and Local Rings}
	Let $\emptyset\neq V\subseteq \mathbb{A}^n$, $\Gamma(V)$ the coordinate ring. Since $\Gamma(V)$ is a domain, may form its quotient field, called the \textbf{field of rational functions }on $V$, and is written $k(V)$.\\
	Let $f\in k(V)$ and $P\in V$, we say that $f$ is defined at $P$ if $f=h/g$for some $h,g\in \Gamma(V)$, $g(P)\neq 0$, in other words, find a "denominator" for $f$ that doesn't vanish at $P$. If $\Gamma(V)$ is UFD, then there is an unique representation $f=h/g$, where $h,g\in\Gamma(V)$ have no common factors, and $f$ is defined at $P$ iff $g(P)\neq 0$.\\
	Fix $P\in V$, define $\mathcal{O}_P(V)$ to be the set of rational functions on $V$ that are defined at $P$. It is easy to verify that $\mathcal{O}_P(V)$ forms a subring of $k(V)$ containing $\Gamma(V)$:
	\begin{center}
		\begin{equation*}
			k\subset \Gamma(V)\subset\mathcal{O}_P(V)\subset k(V)
		\end{equation*}
	\end{center}
	The ring $\mathcal{O}_P(V)$ is called the local ring of $V$ at $P$.\\
	For a rational function $f\in k(V)$, the set of points $P\in V$ where $f$ is not defined is called the \textbf{pole set} of $f$.
	\begin{proposition}
		\leavevmode
		\begin{enumerate}[label=(\arabic*)]
			\item The pole set of a rational function is an algebraic subset of $V$. 
			\item $\Gamma(V)=\bigcap_{P\in V}\mathcal{O}_P(V)$
		\end{enumerate}
	\end{proposition}
	\begin{proof}
		For $f\in k(V)$, $G\in k[X_1,...,X_n]$, $\bar{G}$ denotes the residue class in $\Gamma(V)$. Let $J_f=\lbrace G\in k[X_1,...,X_n] |\bar{G}f\in\Gamma(V) \rbrace$. $J_f$ is an ideal $\triangleleft k[X_1,...,X_n]$ containing $I(V)$, and the points of $V(J_f)$ are exactly those where $f$ is not defined, this proves (1). For (2), if $f\in \bigcap_{P\in V}\mathcal{O}_P(V)$, then $V(J_f)=\emptyset$, since $f$ is defined on every point of $V$. So $J_f=k[X_1,...,X_n]$ by Nullstellensatz, thus $1\cdot f\in \Gamma(V)$, which proves (2).
	\end{proof}
	Suppose $f\in\mathcal{O}_P(V)$, can define the value of $f$ at $P$, written $f(P)$. Consider the kernel $\mathfrak{m}_P(V)$ of the evaluation map $f\mapsto f(P)$, called the maximal ideal of $V$ at $P$. Since $\mathcal{O}_P(V)/\mathfrak{m}_P(V)\cong k$ is a field, and an element $f\in\mathcal{O}_P(V)$ is an unit in $\mathcal{O}_P(V)$ iff $f(P)\neq 0$, therefore $\mathfrak{m}_P(V)=\lbrace \text{non-unit of }\mathcal{O}_P(V) \rbrace$.\\
	\begin{lemma}
		The following conditions on a ring $R$ are equivalent:
		\begin{enumerate}[label=(\arabic*)]
			\item The set of non-unit in $R$ forms an ideal. 
			\item $R$ has a unique maximal ideal that contains every proper ideal of $R$.
		\end{enumerate}
	\end{lemma}
	\begin{proof}
		$(1)\Rightarrow(2):$ Cleary every proper ideal must be contained in the set of all non-units.\\
		$(2)\Rightarrow(1):$ Let $z$ be non-unit, consider the ideal $(z)$ generated by $z$, by assumption this ideal is contained in the unique maximal ideal $\mathfrak{m}$, so $z\in\mathfrak{m}$. Clearly $\mathfrak{m}$ contains no unit, so the non-units form an ideal.
	\end{proof}
	\begin{proposition}
		$\mathcal{O}_P(V)$ is a Noetherian local domain.
	\end{proposition}
	\begin{proof}
		Must show any ideal $I\triangleleft\mathcal{O}_P(V)$ is finitely generated.\\
		By the following fact:
		\begin{framed}
		\paragraph{Fact:}
			For a short exact sequence of $R-$mods:
			\begin{equation*}
				0\rightarrow M_3\rightarrow M_2\rightarrow M_1\rightarrow 0
			\end{equation*}
			TFAE:
			\begin{enumerate}[label=(\arabic*)]
				\item $M_2$ is a Noetherian $R-$mod.
				\item $M_1$ and $M_3$ are Noetherian $R-$mod.
			\end{enumerate}
			\end{framed}
		Since $I(V), k[X_1,...,X_n], \Gamma(V)$ can be viewed as $k[X_1,...,X_n]-$mods, and by Hilberts's Basis Theorem and the above fact, $\Gamma(V)$ is Noetherian.\\
		Consider $I\cap \Gamma(V)$ as an ideal of $\Gamma(V)$, then $I\cap \Gamma(V)$ is generated by $f_1,...,f_r$ in $\Gamma(V)$. Now show these $f_1,...,f_r$ actually generates $I$ in $\mathcal{O}_P(V)$ — for if $f\in \mathcal{O}_P(V)$, then $f=h/g$ for some $h,g\in \Gamma(V)$, and thus $f\cdot g\in \Gamma(V)=\sum a_if_i$, therefore $f=\sum\frac{a_i}{g}f_i$.
	\end{proof}
	\subsection{Discrete Valuation Ring}
	\begin{proposition}
		Let $R$ be a domain that is not a field, then TFAE:
		\begin{enumerate}[label=(\arabic*)]
			\item $R$ is \textbf{Noetherian} and 
			\textbf{local}, and the maximal ideal is \textbf{principal}.
			\item There is an irreducible element $t\in R$ .st. every nonzero $z\in R$ may be written uniquely in the form $z=ut^n$, where $u$ is an unit in $R$, $n\in\mathbb{N}$.
		\end{enumerate}
	\end{proposition}
	\begin{proof}
	Will use the following theorem:
	\begin{framed}
		\begin{theorem*}[Krull Intersection Theorem]
			If $R$ is a local Noetherian ring with maximal ideal $\mathfrak{m}$, then:
			\begin{equation*}
				\bigcap_{n\in\mathbb{N}}\mathfrak{m}^n=\lbrace 0 \rbrace
			\end{equation*}
		\end{theorem*}
	\end{framed}
	$(1)\Rightarrow(2):$ Let $\mathfrak{m}=(m)$. Using Krull Intersection Theorem, there is a unique $n\in\mathbb{N}$ s.t. $z\in(m^n)$ but $z\notin (m^{n+1})$. Write $z=u\cdot m^n$. Since if $u$ is a non-unit, then $u\in (m)$, thus $m^{n+1}\mid z$, so $u$ is a unit, done.\\
	$(2)\Rightarrow (1)$: Easy to see $(t)$ is principal and consits of all non-unit, and all ideals are principal of the form $(t^n)$, so $R$ is PID, hence Noetherian. 
	\end{proof}
	A ring satisfies the above conditions is called a \textbf{discrete valuation ring}, written DVR. An element $t$ as in $(2)$ is called a \textbf{uniformizing parameter} for DVR $R$.\\
	Let $K$ be the fraction field of $R$, then any nonzero $z\in K$ has an unique expression $z=ut^n$, where $u$ is unit and $n\in\mathbb{Z}$, $n$ is called the order of $z$. We define $\text{ord}(0)=\infty$. Note that $R=\lbrace z\in K|\text{ord}(z)\geq0 \rbrace$ and $\mathfrak{m}=\lbrace z\in K|\text{ord}(z)>0  \rbrace$.
	\paragraph{Note:} Consider fraction field $k(X)$ of $k[X]$, the DVR's containing $k$ are:
	\begin{enumerate}
		\item Localization at point $a\in k$: $\mathcal{O}_a=\lbrace F/G\in k(X)|G(a)\neq 0 \rbrace$
		\item Localization "at infinity": $\mathcal{O}_\infty=\lbrace F/G\in k(X)|\text{deg}(G)\geq\text{deg}(F) \rbrace$
	\end{enumerate}
	The second one cannot be expressed as a localization of the form $S^{-1}k[X]$. In fact, it doesn't even contain $k[X]$.\\
	This motivates the concept of projective varieties in Chapter 4.
	\subsection{Forms}
	\begin{definition}[Forms]
		A polynomial $f\in k[X_1,...,X_n]$ is called a \textbf{form} of degree $d$ if:
		\begin{equation*}
			f(\lambda x_1,...,\lambda x_n)=\lambda^d f(x_1,...,x_n)
		\end{equation*}
		i.e. every monomial in $f$ is of degree $d$.
	\end{definition}
	Let $R$ be a domain, if $F\in R[X_1,...,X_{n+1}]$ is a form, we define $F_*\in R[X_1,...,X_n]$ by letting $F_*=F(X_1,...,X_n,1)$. Conversely, for any polynomial $f\in R[X_1,...,X_n]$ of degree $d$, write $f=f_0+f_1+...+f_d$, where $f_i$ is a form of degree $i$, then define $f^*\in R[X_1,...,X_n,X_{n+1}]$ by setting:
	\begin{equation*}
		f^*=X^{d}_{n+1}f_0+X^{d-1}_{n+1}f_1+...+f_d=X^{d}_{n+1}f(X_1/X_{n+1},...,X_n/X_{n+1})
	\end{equation*}
	Then $f^*$ is a form of degree $d$. These processes are described as "dehomogenizing" and "homogenizing" polynomials with respect to $X_{n+1}$.\\
	\begin{proposition}
		\leavevmode
		\begin{enumerate}[label=(\arabic*)]
			\item $(FG)_*=F_*G_*$; $(fg)^*=f^*g^*$.
			\item If $F\neq 0$ and $r$ is the highest power of $X_{n+1}$ that divides $F$, then $X^{r}_{n+1}(F_*)^*=F$; $(f^*)_*=f$.
			\item $(F+G)_*=F_*+G_*$; $X^{t}_{n+1}(f+g)^*=X^{r}_{n+1}f^*+X^{s}_{n+1}g^*$, where $r=\text{deg}(g),\, s=\text{deg}(f),$ and $t=r+s-\text{deg}(f+g)$.
		\end{enumerate}
	\end{proposition}
	\begin{corollary}
		Up to powers of $X_{n+1}$, factoring a form $F\in R[X_1,...,X_{n+1}]$ is the same as factoring $F_*\in R[X_1,...,X_n]$. In particular, if $F\in k[X,Y]$ is a form, where $k$ is algebraically closed, then $F$ factors into pa product of linear factors.\\
	\end{corollary}
	\begin{example*}
		Let $F\in \mathbb{C}[X,Y]$, then $F_*\in \mathbb{C}[X]$ factors into linears factors $(X-a_i)$, and it follows that $\prod (X-a_i Y)=F\in \mathbb{C}[X,Y]$.
	\end{example*}
	\subsection{Ideals with a Finite Number of Zeros}
	The proposition of this section will be used to relate local questions (in terms of the local rings $\mathcal{O}_P(V)$) with global ones (in terms of coordinate rings).
	\begin{proposition}
		Let $I$ be an ideal in $k[X_1,...,X_n]$, $k$ is algebraically closed, and suppose $V(I)=\lbrace P_1,...,P_N \rbrace$ is finite. Let $\mathcal{O}_i=\mathcal{O}_{P_i}(\mathbb{A}^n)$. Then there is a natural isomorphism of $k[X_1,...,X_n]/I$ with $\prod^{N}_{i=1}\mathcal{O}_i/I\mathcal{O}_i$.
	\end{proposition}
	The proof is omitted.
	\begin{corollary}
	$\text{dim}_k(k[X_1,...,X_n]/I)=\sum_{i=1}^{N}\text{dim}_k(\mathcal{O}_i/I\mathcal{O}_i)$
	\end{corollary}
	\begin{corollary}
		If $V(I)=\lbrace P \rbrace$, then $k[X_1,...,X_n]/I$ is isomorphic to $\mathcal{O}_P(\mathbb{A}^n)/I\mathcal{O}_P(\mathbb{A}^n)$.
	\end{corollary}
	
	\section{LOCAL PROPERTIES OF PLANE CURVES}
	\subsection{Multiple Points and Tangent Lines}
	When considering an affine plane curve, for some purposes it is useful to allow $F$ to have multiple factors.\\
	If $F=\prod F_{i}^{e_i}$, where $F_i$ are irreducible factors, we call $e_i$ the \textbf{multiplicity} of the \textbf{component} $F_i$. $F_i$ is said to be \textbf{simple} if $e_i=1$, and \textbf{multiple} otherwise. The components can be recovered by $V(F)$, but the multiplicities cannot.\\
	A point $P$ on the curve $F$ is called a \textbf{simple point} of $F$ if either derivative $F_X(P)=\frac{\partial F(P)}{\partial X}$ or $F_Y(P)=\frac{\partial F(P)}{\partial Y}$ is not 0. In this case the tagent line to $F$ at $P$ is $F_X(P)(X-a)+F_Y(P)(X-b)=0$. A point that isn't simple is called \textbf{multiple} or \textbf{singular}. A curve with only simple points is called a \textbf{nonsingular curve}.
	
	
	
	\begin{figure}[htbp]
		\centering
		
		\begin{minipage}[b]{0.32\textwidth}
			\centering
			\includegraphics[width=\textwidth]{cache/curve_A_R2.png}\\
			\textbf{A:} $y^2 - x^3 + x = 0$
		\end{minipage}\hfill
		\begin{minipage}[b]{0.32\textwidth}
			\centering
			\includegraphics[width=\textwidth]{cache/curve_B_R2.png}\\
			\textbf{B:} $y^2 - x^3 = 0$
		\end{minipage}\hfill
		\begin{minipage}[b]{0.32\textwidth}
			\centering
			\includegraphics[width=\textwidth]{cache/curve_C_R2_zoom.png}\\
			\textbf{C:} $(x^2+y^2)^2 + 3x^2y - y^3 = 0$
		\end{minipage}
		
		\caption{Some algebraic curves in $\mathbb{R}^2$}
	\end{figure}
	
	Let $F$ be any curve, $P=(0,0)$, wirte $F=F_m+F_{m+1}+...+F_n$, where $F_i\neq 0$ is a form of degree $i$. We call $m$ the \textbf{multiplicity} of $F$ at $P$, and call $P$ a simple, double, triple, ... point if $m=1,2,3...$ . For other $P=(a,b)$, just replace $F(X,Y)$ by $F^\prime(X^\prime,Y^\prime)$, where $X^\prime=X-a, \, Y^\prime =Y-b$. \\
	Since $F_m$ is a form in two variables, by Corollary 2.8, $F_m$ can be decomposed into linear factors $F_m=\prod L_{i}^{r_i}$, where $L_i=\alpha X+\beta Y$ are tangent lines to $F$ at $P$.\\
	$r_i$ is the \textbf{multiplicity} of the tangent, $L_i$ is called simple, double, ... if $r_1=1,2,...$ . \\
	If $F$ has $m$ distince tangents at $P$ (i.e. all tangents are simple), then $P$ is said to be a \textbf{ordinary multiple point} of $F$. An ordinary double point is called a \textbf{node}. \\
	In figure 1, a calculation of derivatives shows that A is nonsingular, and B,C are singular, where $(0,0)$ is the only singular point on B,C. Notice that the lowerst order form in A,B,C has degree $1,2,3$ respectively. (0,0) is an ordinary multiple point of C with 3 distinct tangent, but is not an ordinary multiple point of B.\\
	If $F=\prod F_{i}^{e_i}$, then $m_P(F)=\sum e_im_P(F_i)$; and if $L$ is a tangent line to $F_i$ with multiplicity $r_i$, for $i=1,2,...$ , then $L$ is tangent to $F$ with multiplicity $\sum e_ir_i$.
	\subsection{Multiplicities and Local Rings}
	Let $F\in k[X_1,...,X_n]$ be irreducible. We can find the multiplicity of $P$ on $F$ in terms of the local ring $\mathcal{O}_P(F)$. In this section we denote the residue class of $G\in k[X_1,...,X_n]$ in $\Gamma(F)=k[X_1,...,X_n]/(F)$ by $\bar{G}$.
	\begin{theorem}
		$P$ is a simple point of $F$ iff $\mathcal{O}_P(F)$ is a discrete valuation ring. In this case, if $L$ is any line through $P$ that is not tangent to $F$ at $P$, then $\bar{L}\in\mathcal{O}_P(F)$ is a uniformizing parameter.
	\end{theorem}
	\begin{proof}
		By making a affine change of coordinates, we may assume WLOG that $L=X$, $Y$ is the tangent line, and $P=(0,0)\in F$. Then it suffices to show that $\mathfrak{m}_P(F)$ is generated by $\bar{X}$.\\
		Note that regardless of the multiplicity, $\mathfrak{m}_P(F)=(\bar{X},\bar{Y})$.\\
		By above assumption, $F=Y+\text{(terms of order }\geq 2)$, and can write $F=YG+X^2H$, where $G=1+\text{(terms of order }\geq 1)$. Oberserve that $G(0,0)=1$, so $\bar{G}\in \Gamma(F)\subseteq\mathcal{O}_P(F)$ is a unit in $\mathcal{O}_P(F)$. 
		By abuse of notation, let $\bar{G}^{-1}$ be the inverse, then $\bar{YG}=\bar{X^2H}$, so $\bar{Y}=\bar{X^2HG^{-1}}$ in $\mathcal{O}_P(F)$, thus $\mathcal{m}_P(F)=(\bar{X},\bar{Y})=(\bar{X})$.\\
		Now suppose $\mathcal{O}_P(F)$ is DVR, define a valuation function $\text{ord}_P$ on $\mathcal{O}_P(F)$. Suppose $L$ is a line through $P$, if $L$ is tengent to $F$ at $P$, then $\text{ord}_P(L)>1$ (recall that $\text{ord}_P(0)=\infty$), and $\text{ord}_P(L)=1$ if $L$ is tengent to $F$ at $P$, for since $L$ passing $P$, $L$ is nonunit in $\mathcal{O}_P(F)$, and by the same argument on the previous part, $L$ is a uniformizing parameter, hence $\text{ord}_P(L)=1$. The rest of the proof follows from the next therem.
	\end{proof}
	\begin{theorem}
		Let $P$ be a point on an irreducible curve $F$, then for all sufficiently large $n$:
		\begin{equation*}
			m_P(F)=\text{dim}_k(\mathfrak{m}_P(F)^n/\mathfrak{m}_P(F)^{n+1})
		\end{equation*}
		So the multiplicity of $F$ at $P$ depends only on the local ring $\mathcal{O}_P(F).$
	\end{theorem}
	\begin{proof}
		Write $\mathcal{O}=\mathcal{O}_P(V),\mathfrak{m}=\mathfrak{m}_P(V)$ for convenience.\\
		From the exact sequence of $k[X,Y]-$mods:
		\[ 0\rightarrow \mathfrak{m}^n/\mathfrak{m}^{n+1}\rightarrow \mathcal{O}/\mathfrak{m}^{n+1}\rightarrow \mathcal{O}/\mathfrak{m}^n  \rightarrow 0 \]
		It follows that it is enough to show $\text{dim}_k(\mathcal{O}/\mathfrak{m}^n)=n\cdot m_P(F)+s$ for some constant $s$ and all $n\geq m_P(F)$. WLOG may assume $P=(0,0)$, so $\mathfrak{m}=(X,Y)\mathcal{O}$.\\
		By the following fact:
		\begin{mdframed}[frametitle={Fact:}]
			\[
			S^{-1}R / S^{-1}I \cong S^{-1}(R/I),
			\]
			where $S^{-1}R$ is the localization of $R$ at $S$, $S$ is a multiplicative subset, $I\cap S= \emptyset   $.
		\end{mdframed}
		We have:
		\[ \mathcal{O}/\mathfrak{m}^n\cong S^{-1}\Gamma(V)/S^{-1}(X,Y)^n \cong S^{-1}( k[X,Y]/(F,(X,Y)^n))\] 
		For $k[X,Y]/(F,(X,Y)^n)$ is already local at the point $(0,0)$, $S^{-1}(k[X,Y]/(F,(X,Y)^n))\cong k[X,Y]/(F,(X,Y)^n)$ 
		, so the problem is reduced to calculating the dimension of $k[X,Y]/(F,(X,Y)^n)$. Since the multiplicity $m=m_P(F)$ is the smallest degree of monomial terms, it follows that $F(X,Y)^{n-m}\subseteq I^n$, so we have a $k$-linear map:
		\[
		\begin{array}{r c c c}
			\psi\colon & k[X,Y]/(X,Y)^{n-m} & \longrightarrow & k[X,Y]/(X,Y)^n \\
			& \bar{G}        & \longmapsto     & \overline{FG}
		\end{array}
		\]
		Now consider the short exact sequence:
		\[ 0\rightarrow k[X,Y]/(X,Y)^{n-m}\xrightarrow{\psi}k[X,Y]/(X,Y)^n\xrightarrow{\pi}k[X,Y]/(F,(X,Y)^n)\rightarrow 0 \]
		Since the $k-$dimension of $k[X,Y]/(X,Y)^n$ is $ \sum_{k=1}^{n-1}k=\frac{n(n-1)}{2}$, it follows the $k-$dimension of $k[X,Y]/(F,(X,Y)^n)$ is: \[\frac{n(n-1)}{2}-\frac{(n-m)(n-m-1)}{2}=nm-\frac{m(m-1)}{2}\]
		for all $n\geq m$.
	\end{proof}
	Note that if $\mathcal{O}_P(F)$ is DVR, then Theorem 3.2 implies that $m_P(F)=1$, this complete the proof of Theorem 3.1.
	\subsection{Intersection Numbers}
	We want to define for plane curves $F,G$ and point $P\in \mathbb{A}^2$ the intersection number $\text{int}_P(F,G)$. We shall first list several properties we want this intersection number to have before given the definition.\\
	We say $F,G$ intersect \textbf{properly} at $P$ if $F,G$ have no common component that passes through $P$, i.e. there is no irreducible $H\in k[X,Y]$ s.t. $P\in H$ and $H\mid F,G$.\\
	\begin{enumerate}[label=(\arabic*)]
		\item $\text{int}_P(F,G)$ is a nonnegative integer if $F,G$ intersect properly at $P$. $\text{int}_P(F,G)=\infty$ if $F,G$ intersect but not properly at $P$ (if $F,G$ share a irr component containing $P$).
		\item $\text{int}_P(F,G)=0$ iff $P\notin F\cap G$. $\text{int}_P(F,G)$ depends only on the components of $F,G$ passing through $P$.
		\item If $T$ is a affine change of coordinates on $\mathbb{A}^2$ and $T(Q)=P$, then $\text{int}_P(F,G)=\text{int}_Q(F^T,G^T)$.
		\item $\text{int}_P(F,G)=\text{int}_P(G,F)$.
	\end{enumerate}
	Two curves $F,G$ are said to intersect transversally at $P$ if $P$ is a simple point both on $F,G$ and the tangent line to $F,G$ at $P$ are different.
	\begin{enumerate}[start=5,label=(\arabic*)]
		\item $\text{int}_P(F,G)\geq m_P(F)m_P(G)$, with equality occurring iff $F,G$ have no common tagent at $P$.
		\item If $F=\prod F_{i}^{r_i},\,F=\prod G_{j}^{s_j}$, then $\text{int}_P(F,G)=\sum_{i,j}\text{int}_P(F_i,G_j)$.
	\end{enumerate}
	The last requirement says that the intersection number should depend only on the image of $G$ in $\Gamma(F)$.
	\begin{enumerate}[start=7,label=(\arabic*)]
		\item $\text{int}_P(F,G)=\text{int}_P(F,G+AF)$ for any $A\in k[X,Y]$.
	\end{enumerate}
	\begin{theorem}
		There is a unique intersection number defined for all plane curve $F,G$ and point $P$ satisfying properties (1)-(7):
		\[ \text{int}_P(F,G)=\text{dim}_k(\mathcal{O}_P(\mathbb{A}^2)/(F,G)) \]
	\end{theorem}
	The proof is omited.\\
	Property (7) is very useful for calculating intersection numbers, for example, $\text{int}_{(0,0)}(D,E)$ can be reduced to $2\text{int}_{(0,0)}(D,Y)+\text{int}_{(0,0)}(D,H)$, where $H=5X^2-3Y^2+4Y^3+4X^2Y$, and $D,E$ as belows:
\begin{figure}[H]
	\centering
	
	\begin{minipage}[b]{0.32\textwidth}
		\centering
		\includegraphics[width=\textwidth]{cache/curve1_X2Y2_sq_3X2Y_Y3.png}\\
		\textbf{D:} $(X^2+Y^2)^2+3X^2Y-Y^3$
	\end{minipage}
	\hspace{0.06\textwidth}
	\begin{minipage}[b]{0.32\textwidth}
		\centering
		\includegraphics[width=\textwidth]{cache/curve2_X2Y2_cu_4X2Y2.png}\\
		\textbf{E:} $(X^2+Y^2)^3-4X^2Y^2$
	\end{minipage}
	
\end{figure}

	\section{PROJECTIVE VARIETIES}
	
\begin{figure}[h]
	\centering
	\begin{minipage}{0.45\textwidth}
		\centering
		\includegraphics[width=\textwidth]{cache/cone_stereo2_sml.png}
	\end{minipage}\hfill
	\begin{minipage}{0.45\textwidth}
		\centering
		\includegraphics[width=\textwidth]{cache/mxcO0.png}
	\end{minipage}
\end{figure}
	\subsection{Projective Space}
	Consider two plane curve $Y^2=X^2+1$ and $Y=\alpha X$, $\alpha \in k$.If $\alpha\neq \pm 1$, they intersect in two points. When $\alpha=\pm 1$, they do not intersect, but the curve is asymptotic to the line. We want to enlarge the plane in such way that two such curves intersect "at infinity".
\begin{figure}[H]
		\centering
		\includegraphics[width=\textwidth]{cache/hyperbola_asymptotes_line.png}
\end{figure}
	To achieve this, we introduce the notion of \textbf{Projective n-space}:
	\begin{definition*}[Projective $n-$space]
		Let $k$ be any field. The \textbf{Projective n-space} over $k$, written $\mathbb{P}^n(K)$ is defined to be the set of all lines through $(0,0,...,0)$ in $\mathbb{A}^{n+1}(k)$. Notice that $(a_1,...,a_{n+1})\neq 0$ and $(\lambda a_1,...,\lambda a_{n+1})$ determine the same line if $\lambda\neq 0$, so an element in $\mathbb{P}^n(k)$ is a equivalence class, or equivalently, can be viewed as points on a semishpere with boundary.
	\end{definition*}
	\begin{figure}[H]
		\centering
		\includegraphics[width=\textwidth]{cache/"ChatGPT Image Dec 29, 2025, 02_57_09 AM.png"}
	\end{figure}
	Elements of $\mathbb{P}^n$ will be called points, a point $P$ determined by $(x_1,...,x_{n+1})\neq 0$ is denoted by $P=[x_1,...,x_{n+1}]$. We say that $(x_1,...,x_{n+1})$ are \textbf{homogeneous coordinates} for $P$. Obviousely these coordinates are not unique, however if $x_i\neq 0$, $P$ can be written uniquely by:
	\[ P=[x_1:x_2:...:x_{i-1}:1:x_{i+1}:...:x_{n+1}] \]
	The coordinates $(x_1,...,x_{i-1},x_{i+1},...,x_{n+1})$ are called the \textbf{nonhomogeneous coordinates} for $P$ w.r.t. $U_i\coloneqq \lbrace [x_1:...:x_{n+1}]\in \mathbb{P}^n | x_i\neq 0 \rbrace$

	One can view $\varphi^{-1}_i:U_i \rightarrow \mathbb{A}^n$, the obviouse bijection, as the projection of (two-sided) ray onto the $i-$th coordinate hyperplane.\\
	We focus on $U_{n+1}$, and let $H_\infty=\mathbb{P}^n-U_{n+1}$, called the \textbf{hyperplane at infinity}. The correspondence $[x_1:...:x_n:0]\leftrightarrow [x_1:...:x_n]$ shows that $H_\infty$ may be identified with $\mathbb{P}^{n-1}$. Note that we can decomposite $\mathbb{P}^n$ into the disjoint union $\mathbb{A}^{n}\sqcup\mathbb{A}^{n-1}\sqcup...\sqcup\mathbb{A}^{0}$.\\
	Points at infinity $H_\infty$ are not intrinsic to projective space;
	they arise only after choosing an affine chart.
	\subsection{Projective Algebraic Sets}
	A point $P\in \mathbb{P}^n$ is said to be a zero of a polynomial $F\in k[X_1,...,X_n]$ if \[ F(x_1,...,x_{n+1})=0 \]
	for \textbf{every choice} of homogeneous coordinates $ (x_1,...,x_{n+1})$ for $P$, write $F(P)=0$. If $F$ is a form, then it suffices to check one representative of $P$. The definitions of projective $V(I), I(X)$ are similar to affine case.
	\begin{definition}
		An ideal $I\in k[X_1,...,X_{n+1}]$ is called \textbf{homogeneous} if for every $F=\sum_{i=0}^{m}F_i \in I$ , $F_i$ is a form of degree $i$, we have also $F_i\in I$.
	\end{definition}
	For $F=\sum_{i=0}^{m}F_i\in k[X_1,...,X_{n+1}]$, and $[x_1:...:x_{n+1}]\in V(F)$. Let $v=(x_1,...,x_{n+1})$, then $F(\lambda v)=\sum_{i=0}^{m}F_i(\lambda v)=\sum_{i=0}^{m}\lambda^i F_i(v)$. So $F$ define uniquely a polynomial $\tilde{F}_\lambda $ in $k[\lambda]$. Since $F$ vanishes on any representative, $\tilde{F}_\lambda$ must be a zero polynomial, hence $F_i(v)=0$ for all $i$ and thus $V(F)=\bigcap_i V(F_i)$.
	\begin{proposition}
		An ideal $I\in k[X_1,...,X_n]$ is homogeneous iff it is generated by a finite number of forms.
	\end{proposition}
	\begin{proof}
		Let $I=(F_1,...,F_r)$ (Hilbert basis), then $I$ is generated by $F_i^{(j)}$, the homogeneous component of $F_i$ of degree $j$.\\
		Clearly $I$ is generated by $F_{i}^{(j)}$. Conversely, let $S=\lbrace F^{(\alpha)} \rbrace$ be a set of forms generating $I$. For $F=F_m+F_{m+1}+...+F_r$ in $I$, it suffices to show $F_m$ is in $I$, then induction complete the proof.
	\end{proof}
	An algebraice set $V\in \mathbb{P}^n$ is \textbf{irreducible} if it is not the union of two smaller algebraic sets. $V$ is irreducible iff $I(V)$ is prime.\\
	We have the following relation:
	\[\lbrace \text{homogeneous ideals in }k[X_1,...,X_{n+1}] \rbrace \overset{V}{\underset{I}{\rightleftarrows}} \lbrace \text{algebraic sets in }\mathbb{P}^n(k) \rbrace\]
	When writing $I(X)$ or $V(I)$, we need to distinguish between affine case and projective case. If necessary, we write $V_p,\,I_p$ for projective oprations, and $V_a,\, I_a$ for affine ones. 
	\paragraph{Cones}
	If $V$ is an algebraic set in $\mathbb{P}^n$, we define the \textbf{cone} over $V$ by:
	\[ C(V)=\lbrace (x_1,...,x_{n+1})\in \mathbb{A}^{n+1} | [x_1:...:x_{n+1}] \in V \text{ or }(x_1,...,x_{n+1})=0  \rbrace \]
	If $V\neq \emptyset$, then $I_a(C(V))=I_p(V)$. If $I$ is a homogeneous ideal in $k[X_1,...,X_{n+1}]$ s.t. $V_p(I)\neq \emptyset$, then $C(V_p(I))=V_a(I)$. These relations reduces many questions about projective space to affine case, for example:
	\begin{theorem}[PROJECTIVE NULLSTELLENSATZ]
		Let $I$ be a homogeneous ideal in $k[X_1,...,X_{n+1}]$. Then:
		\begin{enumerate}[label=(\arabic*)]
			\item $V_p(I)\neq \emptyset$ iff there is an integer $N$ s.t. $I$ contains all forms of degree $\geq N$.
			\item If $V_p(I)\neq \emptyset$, then $I_p(V_p(I))=\text{Rad}(I)$.
		\end{enumerate}
	\end{theorem}
	Let $V$ be a nonempty projective variety in $\mathbb{P}^n$, then $I(V)$ is a prime ideal, so the residue $\Gamma_h(V)-k[X_1,...,X_{n+1}]/I(V)$ is a domain, called the \textbf{homogeneous coordinate ring} of $V$. In contrast to affine case, an element in $\Gamma_h(V)$ does not define a function on $V$ except for constants, since a well-defined function on $V$ must be independent from the choices of homogeneous coordinates.
	\begin{proposition}
		Let $I$ be a homogeneous ideal in $k[X_1,...,X_{n+1}]$ and $\Gamma = k[X_1,...,X_{n+1}]/I$, then every element $f\in \Gamma$ can be written uniquely as $f=f_0+...+f_m$, where $f_i$ is a form of degree $i$.
	\end{proposition}
	\begin{scratch}
		Existence is trivial. For uniqueness, let $F,G\in k[X_1,...,X_{n+1}]$ have the same residue, write them as a sum of forms, subtract one from the other, the homogeneity of $I$ implies that each form after subtraction is in $I$.
	\end{scratch}
	Let $k_h(V)$ be the quotient field of $\Gamma_h(V)$. Most non-constant element in $k_h(V)$ are not functions on $V$. However if $f,g$ are forms in $\Gamma_h(V)$ of the same degree, then $f/g$ does define a function on $g\neq 0$.\\
	Define $k(V)=\lbrace z\in k_h(V)| z=f/g \text{ for some }f,g\in \Gamma_h(V) \text{ of the same degree} \rbrace$. Elements of $k(V)$ are called rational functions on $V$.\\
	Now we can define the local ring $\mathcal{O}_P(V)$ of $V$ at a point $P$ as in affine case, the maximal ideal is \[ \mathfrak{m}_P(V)=\lbrace z|z=f/g,\, g(P)\neq 0 = f(P) \rbrace \]
	\\
	If $T:\mathbb{A}^{n+1}\rightarrow \mathbb{A}^{n+1}$ is a linear change of coordinates, then $T$ takes lines through origin into lines through origin, thus $T$ determines a map from $\mathbb{P}^n$ to $\mathbb{P}^n$, called a projective change of coordinates.\\
	The results of affine changes of coordinates also apply here.\\
	Note that in projective geometry, one performs projective linear transformations on the whole projective space, without fixing a hyperplane at infinity.
	The notion of points at infinity only appears after choosing an affine chart.
	\subsection{Affine and Projective Varieties}
	By the map $\varphi_{n+1}:\mathbb{A}^n\rightarrow U_{n+1}\subset \mathbb{P}^n$, consider $\mathbb{A}^n$ as a subset of $\mathbb{P}^n$. We want to study the relations between the algebraic sets in $\mathbb{A}^n$ and those in $\mathbb{P}^n$.\\
	This transformation is carried out via homogenization, as discussed in Chapter 2, Section 6.\\
	Let $V$ be an algebraic set in $\mathbb{A}^n$, $I=I_a(V)\subset k[X_1,...,X_n]$. Let $I^*$ be the ideal in $k[X_1,...,X_{n+1}]$ generated by $\lbrace F^*|F\in I \rbrace$. Also define $V^*$ to be $V_p(I^*)\subset \mathbb{P}^n$.\\
	Conversely, let $V$ be an algebraic set in $\mathbb{P}^n$, $I=I_p(V)\subset k[X_1,...,X_{n+1}]$. Let $I_*$ be the ideal in $k[X_1,...,X_n]$ generated by $\lbrace F_*|F\in I \rbrace$, and define $V_*$ to be $V(I_*)\subset \mathbb{A}^n$.
	
	\begin{proposition}
		\begin{enumerate}[label=(\arabic*)]
			\item If $V\subset \mathbb{A}^n$, then $\varphi_{n+1}(V)=V^*\cap U_{n+1}$, and $(V^*)_*=V$. 
			\item The operators $(\cdot)^*$ and $(\cdot)_*$ preserve inclusions. More precisely, if $V\subset W\subset \mathbb{A}^n$, then $V^*\subset W^*\subset \mathbb{P}^n$; and if  $V\subset W\subset \mathbb{P}^n$, then $V_*\subset W_*\subset \mathbb{A}^n$.
			\item If $V$ is irreducible in $\mathbb{A}^n$, then $V^*$ is irreducible in $\mathbb{P}^n$.
			\item If $V=\bigcup_i V_i$ is the irreducible decomposition of $V$ in $\mathbb{A}^n$, then $V^*=\bigcup_i V^{*}_{i}$ is the irreducible decomposition of $V^*$ in $\mathbb{P}^n$.
			\item If $V\subset \mathbb{A}^n$, then $V^*$ is the smallest algebraic set in $\mathbb{P}^n$ that contains $\varphi_{n+1}(V)$.
			\item If $V\subsetneq \mathbb{A}^n$ is not empty, then no component of $V^*$ lies in or contains $H_\infty$.
			\item If $V\subset \mathbb{P}^n$, and no component of $V$ lies in or contains $H_\infty$, then $V_*\subsetneq \mathbb{A}^n$ and $(V_*)^*=V$.
		\end{enumerate}
		\end{proposition}
		If $V$ is an algebraic set in $\mathbb{A}^n$, $V^*\subset \mathbb{P}^n$ is called the \textbf{projective closure} of $V$. Except for projective varieties lying in $H_\infty$, there is a natural one-to-one correspondence between nonempty affine and projective varieties.\\
		The below figure shows an example, consider affine variety $V(X^2-Y^2=1)$, an hyperbola on $\mathbb{A}^2$. To find its projective closure, we homogenize the defining equation and obtain $(X^2-Y^2-1)^*=(X^2-Y^2-Z^2)\in k[X,Y,Z]$. Now $V_p(X^2-Y^2-Z^2)$ is the projective closure of the original hyperbola. Note that the variety $V_a(X^2-Y^2-Z^2)$ in $\mathbb{A}^3$ is a cone with $X-$axis as its axis, and the projection onto $Z=1$ plane recovers the original hyperbola.
		
		\begin{figure}[h]
			\centering
			\includegraphics[width=0.45\textwidth]{cache/圓錐截線_雙曲線.png}
			\caption{The Z-axis is directed rightward, X-axis is directed upward.}
		\end{figure}
		Note that the projective closure of any nondegenerate conic section, such as a circle, ellipse, parabola, or hyperbola, is a smooth projective conic.
		After a suitable projective linear change of coordinates, it can be written in the form:
		\[ X^2+Y^2=Z^2 \]
		Let $V$ be affine variety, $V^*$ its projective closure. If $f\in \Gamma_h(V^*)$ is a form of degree $d$, define $f_*$ as follows: take a form $F\in k[X_1,...,X_{n+1}]$ whose $I_p(V^*)$-residue is $f$, and let $f_*=I(V)-$residue of $F_*$. One checks this definition is independent of choices of $F$.\\ Then we have a natural isomorphism $\alpha: k(V^*)\rightarrow k(V)$ s.t. $\alpha(f/g)=f_*/g_*$, where $f,g$ are forms of the same degree on $V^*$. For $P\in V$, can consider $P\in V^*$ via $\varphi_{n+1}$, then $\alpha$ induces an isomorphism of $\mathcal{O}_P(V^*)$ with $\mathcal{O}_P(V)$.

	\subsection{Multiprojective Space}
	In this section we discuss the product of projective spaces and varieties.\\
	First we discuss the product of affine spaces and varieties.
	\begin{center}
	\begin{tabular}{|c|c|c|c|}
		\hline
		Space& $\mathbb{A}^m$ & $\mathbb{A}^n$ & $\mathbb{A}^{m+n}$ \\ \hline
		
		Point &   $(x_1,...,x_m)$    & $(y_1,...,y_n)$      &      $(x_1,...,x_m,y_1,...,y_n)$ \\\hline
		Coordinate ring &   $k[X_1,...,X_m]$    &  $k[Y_1,...,Y_n]$     & $k[X_1,...,X_m,Y_1,...,Y_n]$       \\\hline
		Variety &   $V_1$    &  $V_2$     &    $V_1\times V_2$   \\\hline
		Ideal &   $I_1$    &   $I_2$    &    $(I_1,I_2)$   \\ \hline
	\end{tabular}
	\end{center}
	In the projective case, some restrictions must be made, since points are defined only up to scalar multiplication.\\
	Consequently, the product $\mathbb{P}^m\times \mathbb{P}^n$ is not $\mathbb{P}^{m+n}$. For example, $\mathbb{P}^1\times \mathbb{P}^1$ is not $\mathbb{P}^2$. In fact this isn't true topologically, for $\mathbb{P}^1\times \mathbb{P}^1\cong S^1\times S^1$ is orientable (i.e. one can choose orientations locally in a globally consistent way), but $\mathbb{P}^2$ is not.\\
	Write $X=(X_1,...,X_{n+1}),\, Y=(Y_1,...,Y_{m+1})$ and $k[X,Y]$. A polynomial $F\in k[X,Y]$ is called a \textbf{biform} of \textbf{bidegree} $(p,q)$ if $F$ is a from of degree $p$ in $X$, and a form of degree $q$ in $Y$, respectively.\\
	A ideal $I\in k[X,Y]$ is called \textbf{bihomogeneous} if every $F\in I$, and component of any fixed bidegree is in $I$. Like Prop 4.2, an ideal in $k[X,Y]$ is bihomogeneous iff it is generated by a set of biforms.\\
	Now we can define algebraic sets and varieties in the product projective space $\mathbb{P}^n\times \mathbb{P}^m$ with the above definition in the obvious way. Note that a product of varieties $V_1\times V_2$ is $V(I_1,I_2)$, where $I_1=I_{\mathbb{P}^n}(V_1)$, $I_2=I_{\mathbb{P}^m}(V_2)$.
	\paragraph{Segre Embedding}
	The product of projective spaces can be realized as a subvariety of a higher-dimensional projective space. This embedding is called \textbf{Segre embedding}, defined as follows:
	\begin{align*}
		f:\mathbb{P}^m \times \mathbb{P}^n
		&\longrightarrow \mathbb{P}^{mn+m+n} \\
		\bigl( (x_1,\dots,x_{m+1}), (y_1,\dots,y_{n+1}) \bigr)
		&\longmapsto
		(x_1y_1,\dots,x_{m+1}y_1,\;
		x_1y_2,\dots,x_{m+1}y_2,\;
		\dots,\;
	x_{m+1}y_{n+1})
	\end{align*}
	Consider the image of $f$. It is a projective variety in $\mathbb{P}^{mn+m+n}$,
	which is the zero set of the polynomials
	\[
	z_{i,j}z_{k,l} - z_{i,l}z_{k,j},
	\]
	for all $i,j,k,l$, where $z_{i,j} \coloneqq X_i Y_j$.\\
	When studying the product of projective varieties,
	bihomogeneous ideals give the intrinsic algebraic description and are better suited for computations and scheme-theoretic operations;
	while the Segre embedding provides a convenient way to view it as a projective variety and gain geometric intuition. The figure below illustrates the Segre embedding of  $\mathbb{P}^1 \times \mathbb{P}^1$ into $\mathbb{P}^3$.
	\begin{figure}[h]
		\centering
		\includegraphics[width=0.6\textwidth]{cache/GOg1l.png   }
		\caption{Segre embedding of $\mathbb{P}^1\times \mathbb{P}^1$ into $\mathbb{P}^3$}
	\end{figure}
	
%////////////////////////////////////////////////////////////////////////////////////////////////////////////////////////////////////////////////////////////////////////////////////////////////////////////////////////////////////////////////////////////////////////////////////////////////////////////////////////////////////////////////////////////////////////////////////////////////////////%
	
	\section{APPENDIX}
	\subsection{Gröbner basis}
	For polynomials with one variable $f(x),g(X)\in k[X]$, we have polynomial division algorithm $f(X)=g(X)q(X)+r(X)$, where the degree of the remainder $r(X)$ is less than $g(X)$. But for multivariate polynomials, there's no natural ordering of polynomials. So in order to do \textbf{generalized polynomial division}, we need an \textbf{monomial ordering}, usually the lexico order $x_1>x_2>...$ or some permutation $x_{\alpha_1}>x_{\alpha_2}>...$.\\
	We also need another tool called Gröbner basis to perform generalized polynomial division, Gröbner basis is strongly related to some well-known algorithm, such as:
	\begin{itemize}
		\item Gaussian Elimination
		\item Euclidean Algorithm for Computing gcd
		\item Simplex Algorithm
		\item ... 
	\end{itemize}
	\begin{center} 
	\begin{tikzpicture}
		
		\node (X1) at (4,0) {\fbox{Set of input polynomials $F=\lbrace f_1,...,f_n \rbrace$}};
		\node (X2) at (4,-2) {\fbox{Set of output polynomials $G=\lbrace g_1,...,g_m \rbrace$(Gröbner basis)}};

		
		\draw[->] (X1)--(X2) node[midway,right]{Buchberger Algorithm};
	\end{tikzpicture}
	\end{center}
	\begin{center}
			
			\begin{tikzpicture}
			
			\node (X1) at (4,0) {$\lbrace 2x+3y+4z-5,3x+4y+5z-2 \rbrace$:Two plane in $\mathbb{R}^3$};
			\node (X2) at (4,-2) {$\lbrace x-z+14,y+2z-11  \rbrace$:Two line in $x-z,y-z$ plane, respectively};
			
			
			\draw[->] (X1)--(X2) node[midway,right]{Buchberger Algorithm};
		\end{tikzpicture}
	\end{center}
	Now give the definition of Gröbner basis, first fix a lexico ordering $X_1>X_2>...$.
	\begin{definition}[Initial Monomial,Ideal]
		Let $>$ be a monomial ordering, define the \textbf{initial monomial} $in_{>}(F)$ to be the leading term of the polynomial $F\in k[X_1,...,X_n]$ i.e. the monomial in $F$ of maximal degree.\\
		Consider the ideal consists of every initial monomial of polynomial in $I$, called the \textbf{initial ideal}, denoted by $in_>(I)$. Can see $in_>(I)$ is indeed an ideal in $k[X_1,...,X_n]$.
	\end{definition}
	\begin{definition}[Gröbner basis]
		Given polynomials $f_1,...,f_r$ and monomial ordering $>$, a finite set of generators $\lbrace g_1,g_2,...,g_m \rbrace$ of ideal $I=(f_1,...,f_r)\in k[X_1,...,X_n]$ is called a \textbf{Gröbner basis} of $I$ if $in_>(I)=(in_>(g_1),...,in_>(g_m))$.
	\end{definition}
	Will show an combinatory application of Gröbner basis:
	\begin{question}
		Minimize linear function $P+N+D+Q$, while $P,N,D,Q\in\mathbb{N}\cup \lbrace 0 \rbrace$, with constraint $ P+5N+10D+25Q=117$.
	\end{question}
	\textbf{Answer:} Consider polynomial ring $\mathbb{Q}[C,P,N,D,Q]$, can represent a combination of coins by a monomial $C^{n}P^{n_P}N^{n_N}D^{n_D}Q^{n_Q}$, where $n=\sum_\alpha n_\alpha$.\\
	Now define relations:
	\begin{align*}
		&f_1=CN-C^5P^5\\
		&f_2=CD-C^{10}P^{10}\\
		&f_3=CQ-C^{25}P^{25}
	\end{align*}
	These three polynomial represent all the exchange rules of coins, which are:
	\begin{align*}
		&\text{1 nickel }=\text{ 5 pennies}\\
		&\text{1 dime }=\text{ 10 pennies}\\
		&\text{1 quarter }=\text{ 25 pennies}\\
	\end{align*}
	The $C-$degree of each monomial in the relations represents the number of coins.\\
	Now choose an ordering $C> P> N> D> Q$, then apply Buchberger Algorithm to these three polynomials, will get a unique \textbf{reduced Gröbner basis} consists of nine polynomials (computed by program, might be wrong):
	\begin{align*}
		&g_1: C^5P^5 - CN\\
		&g_2: C^2N^2 - CD\\
		&g_3: C^3ND^2 - CQ\\
		&g_4: C^4P^5D - CN^3\\
		&g_5: C^2D^3 - CNQ\\
		&g_6: CN^3Q - CD^4\\
		&g_7: C^3P^5D^2 - CN^5\\
		&g_8: CP^5Q - CN^6\\
		&g_9: CP^5D^4 - CN^9\\
	\end{align*}
	In order to minimize total number of coins, we apply generalized polynomial division to $C^{117}P^{117}$(represents 117 coins of penny), and calculate the remainder, which is:
	\begin{center}
		\fbox{$C^8P^2NDQ^4$}
	\end{center}
	The multi-degree $(8,2,1,1,4)$ means total of 8 coins, with 2 pennies, 1 nickel, 1 dime, and 4 quarters.\\
	Here are some total amount of cents and its minimal number of coins represented with monomials:\\
	\begin{center}
\begin{tabular}{|c|c|}
	\hline
	Dividend & Remainder \\
	\hline
	$C^{30}P^{30}$ & $C^{2}NQ$ \\
	$C^{567}P^{567}$ & $C^{26}P^{2}NDQ^{22}$ \\
	$C^{9999}P^{9999}$ & $C^{405}P^{4}D^{2}Q^{399}$ \\
	$C^{35857}P^{35857}$ & $C^{1437}P^{2}NQ^{1434}$ \\
	$C^{8679031}P^{8679031}$ & $C^{347163}PNQ^{347161}$ \\
	\hline
\end{tabular}
\end{center}
\textbf{Note:} The additional variable $C$ for counting number of coins is crucial, one may try to work under $\mathbb{Q}[P,N,D,Q]$ and compute Gröbner basis for:
\begin{align*}
	&f^{\prime}_{1}=N-P^5\\
	&f^{\prime}_{2}=D-P^{10}\\
	&f^{\prime}_{3}=Q-P^{25}
\end{align*}
with ordering $P>N>D>Q$, this is the only reasonable choice of lexico order, since generalized polynomial division tend to make the remainder as small as possible, and the least favorite coin is penny, otherwise the remainder of, for example $Q$ may be $P^{25}$, in which case the total number of coins increases. Similarly for $N,D,Q$. But one observes that:
\begin{equation*}
	QN\sim D^3 
\end{equation*}
Which the degree on the left-hand-side is greater, but the right one has much number of coins.
\\
\begin{theorem}[Standard Monomials]
	Let $I\triangleleft k[X_1,...,X_n]$, $>:$ monomial order.\\
	A monomial $X_{1}^{a_1}X_{2}^{a^2}...X_{n}^{a_n}$ is said to be \textbf{standard} if it is not in the initial ideal $in_>(I)$.\\
\end{theorem}
\begin{example*}
	n=3, $in_>(I)=(X_{1}^{3},X_{2}^{4},X_{3}^{5})$, then the standard monomials are $X_{1}^{a_1}X_{2}^{a^2}X_{3}^{a_3}$, where $a_1=0,1,2$, $a_2=0,1,2,3$, $a_3=0,1,2,3,4$, totally $60$ standard monomials.\\
	On the other hand, if  $in_>(I)=(X_{1}^{3},X_{2}^{4},X_1X_{3}^{4})$, then there're infinitely many standard monomials. (Since all $X_{3}^{t}$ are not in $in_>(I)$)
\end{example*}
\begin{theorem}[Fundamental Theorem of Algebra (Generalized)]
	Let $I=(g_1,...,g_m)$, $\lbrace g_i \rbrace$ forms a Gröbner basis under monomial order $>$,\\ then $|V(I)|\text{ (counted with multiplicity)}\text{ is equal to the number of standard monomial w.r.t. }I$.
\end{theorem}
\begin{example*}
	Consider an ideal $I$ generated by Gröbner basis $\lbrace x-2yz+2y+z,y^2+yz+y-z-\frac{2}{3},z^2+z-1 \rbrace$ in $k[x,y,z]$, the leading terms are $x,y^2,z^2$ respectively, so there are $1\cdot 2 \cdot 2=4$ standard monomials, hence by Fundamental Theorem of Algebra, $V(I)=4$.
\end{example*}
Note that the theorem also holds for $|V(I)|=\infty$ or exists infinitely many standard monomials. \\
\\
Solving a system of polynomials involves elimination of variables. We begin by eliminating all polynomials involving any variable $\in \lbrace X_1,...,X_{l-1},X_{l} \rbrace$.\\
\begin{definition}[$l-$Elimination Ideal]
	Let $I\triangleleft k[X_1,...,X_n]$. The \textbf{$l-$th elimination ideal }$I_l$ is the ideal $\triangleleft k[X_{l+1},...,X_n]$ defined by:
	\begin{equation*}
		I_l=I\cap k[X_{l+1},...,X_n]
	\end{equation*}
\end{definition} 
For a fix $l\in \mathbb{N}$ s.t. $1\leq l\leq n$, we say a monomial order $>$ on $k[X_1,...,X_n]$ is of \textbf{$l-$elimination type}, if any monomial invloving any of $X_1,...,X_{l-1},X_l$ is greater than all other monomial in $k[X_{l+1},...,X_n]$.\\
For example, the lexico ordering is of $l-$elimination type.
\begin{theorem}[Elimiation Theorem]
	Let $I\triangleleft k[X_1,...,X_n]$, and $G:$ Gröbner basis of $I$ w.r.t. a $l-$elimination type order. Then:
	\begin{equation*}
		G_l=G\cap k[X_{l+1},...,X_n]
	\end{equation*} 
	is a Gröbner basis of the $l-$elimination ideal $I_l$.
\end{theorem}
\subsection{Simplicies $\&$ Simplicial Complexes}
\textbf{Simplices} are the higher-dimensional analogues of triangles. 
\paragraph{Convex Combination}
\begin{center}
	

	\begin{tikzpicture}[>=Stealth, every node/.style={font=\small}]
		
		% ==================== 左圖 ====================
		\begin{scope}[shift={(0,0)}]
			% axes
			\draw[->] (0,0) -- (4,0) node[below right] {$x$};
			\draw[->] (0,0) -- (0,4) node[above left] {$y$};
			
			\foreach \i in {1,2,3}
			\draw (\i,0.05) -- (\i,-0.05) node[below] {\i};
			\foreach \j in {1,2,3}
			\draw (0.05,\j) -- (-0.05,\j) node[left] {\j};
			
			% vectors
			\coordinate (O) at (0,0);
			\coordinate (v0) at (2.4,1.2);
			\coordinate (v1) at (1.0,2.8);
			
			% vector arrows
			\draw[->] (O) -- (v0);
			\draw[->] (O) -- (v1) node[midway, above left] {$\vec{v_1}$};
			
			% v1 label (below)
			\node[below right] at ($(O)!0.5!(v0)$) {$\vec{v_0}$};
			
			% thick connecting segment
			\draw[very thick] (v0) -- (v1);
			
			% ---- interior point for left figure ----
			\coordinate (pL) at ($(v0)!0.45!(v1)$);
			\fill (pL) circle (0.035);
			
			% curved arrow → pL
			\draw[->, bend left=20, thick]
			($(pL)+(1.0,0.9)$) to (pL);
			
			% label (horizontal)
			\node[above] at ($(pL)+(1.0,0.9)$)
			{$\lambda_0 \vec{v_0} + \lambda_1 \vec{v_1}$};
			
			\fill (0,0) circle (0.03);
		\end{scope}
		
		
		% ==================== 右圖:三角形 + 兩個示意點 ====================
		\begin{scope}[shift={(6,0)}]
			% axes
			\draw[->] (0,0) -- (4.5,0) node[below right] {$x$};
			\draw[->] (0,0) -- (0,4.5) node[above left] {$y$};
			
			\foreach \i in {1,2,3,4}
			\draw (\i,0.05) -- (\i,-0.05) node[below] {\i};
			\foreach \j in {1,2,3,4}
			\draw (0.05,\j) -- (-0.05,\j) node[left] {\j};
			
			% simplex vertices
			\coordinate (O) at (0,0);
			\coordinate (A) at (3.2,0.9);  % v1
			\coordinate (B) at (1.0,3.0);  % v2
			\coordinate (C) at (3.0,2.8);  % v3
			
			% filled triangle
			\fill[gray!30] (A) -- (B) -- (C) -- cycle;
			
			% boundary
			\draw[thick] (A) -- (B) -- (C) -- cycle;
			
			% arrows from origin
			\draw[->] (O) -- (A) node[midway, below right] {$\vec{v_0}$};
			\draw[->] (O) -- (B) node[midway, above left] {$\vec{v_1}$};
			\draw[->] (O) -- (C);
			\node[below] at ($(O)!0.5!(C)$) {$\vec{v_2}$};
			
			% ---- interior point p ----
			\coordinate (p) at ($0.3*(A)+0.4*(B)+0.3*(C)$);
			\fill (p) circle (0.035);
			
			% curved arrow → p
			\draw[->, bend left=20, thick]
			($(p)+(1.2,1.0)$) to (p);
			
			% label (horizontal)
			\node[above] at ($(p)+(1.2,1.0)$)
			{$\lambda_0 \vec{v_0} + \lambda_1 \vec{v_1} + \lambda_2 \vec{v_2}$};
			
			% ---- second point q on AB ----



			
		\end{scope}
		
	\end{tikzpicture}
	


\end{center}
Points in triangle are uniquely determined by $(\lambda_0,...,\lambda_n)$, $\lambda_i\geq 0$ and $\sum_{i=0}^{n}\lambda_i=1$. These $(\lambda_0,...,\lambda_n)$ are called the barycentric coordinates for simplcies.\\
In $\mathbb{A}^n$, let $\vec{v_0},...,\vec{v_n}$ be points not lying on same hyperplane, then 
the $n-$simplex $\Delta_n$, denoted by $(\vec{v_0}, \vec{v_1},...,\vec{v_n})$ is described by:
\begin{equation*}
	\lbrace \lambda_0\vec{v_0}+\lambda_1\vec{v_1}+...+\lambda_n\vec{v_n}|0\leq \lambda_i\leq 1, \sum \lambda_i=1 \rbrace
\end{equation*}
A \textbf{$n-$dimensional standard form} of simplex is the simplex defined by $(e_1,...,e_{n+1})$ in $\mathbb{A}^{n+1}$, which lies in the hyperplane $x_1+x_2+...+x_{n+1}=1$.\\
\paragraph{Simplicial Complexes}
\\\\
Building up a space using simplicies.\\
Any 2 simplicies in a simplicial complex are either disjoint or meet in a common face.\\
A face of a simplicies $(x_1,...x_n)$ is a simplcies whose vertices are in $x_1,...,x_n$. For example $(x_2,x_3,x_4),(x_1,x_3,x_4),(x_1,x_2,x_4),(x_1,x_2,x_3)$ are 4 different 2-dimesional faces, $(x_i,x_j),i\neq j$ are 1-dimensional(edges) faces of $(x_1,x_2,x_3,x_4)$.
\begin{center}
	\begin{tikzpicture}
		\coordinate (A) at (0,0);
		\coordinate (B) at (3,0);
		\coordinate (C) at (0,2);
		\coordinate (D) at (2,3);
		\coordinate (E) at (4,2);
		\coordinate (F) at (5,1);
		\coordinate (G) at (-1.5,1.5);
		\draw (A)--(C);
		\draw (A)--(B);
		\draw (A)--(D);
		\draw[dashed] (B)--(C);
		\draw (B)--(D);
		\draw (C)--(D);
		\draw (B)--(E);
		\draw (D)--(E);
		\draw (B)--(F);
		\draw (E)--(F);
		\draw (G)--(C);
		% 三個重疊的三角形
		\fill[gray, opacity=0.3] (A)--(B)--(C)--cycle;
		\fill[gray, opacity=0.3] (B)--(C)--(D)--cycle;
		\fill[gray, opacity=0.3] (A)--(C)--(D)--cycle;
		\fill[gray, opacity=0.3] (B)--(E)--(F)--cycle;
		
		% 點
		\foreach \p in {A,B,C,D,E,F,G}{
			\fill (\p) circle (2pt);
		}
	\end{tikzpicture}

\end{center}
3-simplex is the tetrahedron, 2-simplcies are all the triangles(including the faces of tetrahedron), 1-simplecies are lines, 0-simplcies are points. 
\paragraph{Orientation}
A simplex is oriented by choosing an ordering of its vertices; two orderings give the same orientation iff they differ by an even permutation. So for any $n-$dimensional simplex $\Delta_n$, there are only two orientation.\\
\paragraph{Boundary}
For a 1-dimensional oriented simplex $\Delta_1=(v_0,v_1)$, define the boundary $\partial\Delta_1=v_1-v_0.$\\
For 2-dimensional $\Delta_2=(v_0,v_1,v_2)$, define \\$\partial \Delta_2=(v_0,v_1)+(v_1,v_2)+(v_2,v_0)=(v_0,v_1)-(v_0,v_2)+(v_1,v_2).$\\
For general $\Delta_n=(v_0,...,v_n)$, we define the boundary $\partial \Delta_n=\sum_{i=0}^{n}(-1)^i(v_0,...,v_{i-1},\hat{v_i},v_{i+1},...,v_n)$,\\
where the hat means 'omit', for example $(v_0,\hat{v_1},v_2)=(v_0,v_2)$.
\paragraph{Simplicial Chain}
Will give an example to show how is chain complex useful.\\
Consider the following diagram of a two triangles sharing an edge, where one trianle colored in gray means it's a 2-simplex:
\begin{center}
	\begin{tikzpicture}
		\coordinate (A) at (0,0);
		\coordinate (B) at (2,1);
		\coordinate (C) at (2,-1);
		\coordinate (D) at (4,0);
		\fill (A) circle (2pt) node[above left] {$A$};
		\fill (B) circle (2pt) node[above] {$B$};
		\fill (C) circle (2pt) node[below] {$C$};
		\fill (D) circle (2pt) node[above right] {$D$};
		
		\draw (A)--(B);
		\draw (A)--(C);
		\draw (C)--(B);
		\draw (D)--(B);
		\draw (D)--(C);
		\fill[gray, opacity=0.3] (A)--(B)--(C)--cycle;
		\foreach \p in {A,B,C,D}{
			\fill (\p) circle (2pt);
		}
	\end{tikzpicture}
\end{center}
Consider chain complex:
\begin{equation*}
	\mathbb{Z}[F]\xrightarrow{\partial_2}\mathbb{Z}[E]\xrightarrow{\partial_1}\mathbb{Z}[V]
\end{equation*}
Where $F,E,V$ are sets of faces, edges, points, repectively. \\Easy to see $\text{Im }\partial_2$ is the $\mathbb{Z}-$linear combination of $(A,B)+(B,C)+(C,A)$, and $\text{Ker} \partial_1$ is the $\mathbb{Z}-$linear combination of $(A,B)+(B,C)+(C,A)$ and $(B,D)+(D,C)+(C,B)$.
\begin{center}
	\begin{tikzpicture}
		\node () at (-3,1) {Im $\partial_2$:};
		\node () at (-1,0) {$\mathbb{Z}\cdot$};
		\coordinate (A) at (0,0);
		\coordinate (B) at (2,1);
		\coordinate (C) at (2,-1);
		
		\fill (A) circle (2pt) node[above left] {$A$};
		\fill (B) circle (2pt) node[above] {$B$};
		\fill (C) circle (2pt) node[below] {$C$};
	
		
		\draw[postaction={
			decorate,
			decoration={markings, mark=at position 0.5 with {\arrow{Stealth}}}
		}
		] (A) -- (B);
		\draw[postaction={
			decorate,
			decoration={markings, mark=at position 0.5 with {\arrow{Stealth}}}
		}
		] (C) -- (A);
		\draw[postaction={
			decorate,
			decoration={markings, mark=at position 0.5 with {\arrow{Stealth}}}
		}
		] (B) -- (C);
		
	\end{tikzpicture}
\end{center}
\begin{center}
	\begin{tikzpicture}
		\node () at (-3,1) {Ker $\partial_1$:};
		\node () at (-1,0) {$\mathbb{Z}\cdot$};
		\coordinate (A) at (0,0);
		\coordinate (B) at (2,1);
		\coordinate (C) at (2,-1);
		
		\fill (A) circle (2pt) node[above left] {$A$};
		\fill (B) circle (2pt) node[above] {$B$};
		\fill (C) circle (2pt) node[below] {$C$};
	
		\draw[postaction={
			decorate,
			decoration={markings, mark=at position 0.5 with {\arrow{Stealth}}}
		}
		] (A) -- (B);
		\draw[postaction={
			decorate,
			decoration={markings, mark=at position 0.5 with {\arrow{Stealth}}}
		}
		] (C) -- (A);
		\draw[postaction={
			decorate,
			decoration={markings, mark=at position 0.5 with {\arrow{Stealth}}}
		}
		] (B) -- (C);
		
		
		\node () at (-1,0) {$\mathbb{Z}\cdot$};
		
		\coordinate (B1) at (6,1);
		\coordinate (C1) at (6,-1);
		\coordinate (D1) at (8,0);
		\node () at (3.5,0) {$+$};
		\node () at (5,0) {$\mathbb{Z}\cdot$};
		\fill (B1) circle (2pt) node[above] {$B$};
		\fill (C1) circle (2pt) node[below] {$C$};
		\fill (D1) circle (2pt) node[right] {$D$};
		
		\draw[postaction={
			decorate,
			decoration={markings, mark=at position 0.5 with {\arrow{Stealth}}}
		}
		] (B1) -- (D1);
		\draw[postaction={
			decorate,
			decoration={markings, mark=at position 0.5 with {\arrow{Stealth}}}
		}
		] (D1) -- (C1);
		\draw[postaction={
			decorate,
			decoration={markings, mark=at position 0.5 with {\arrow{Stealth}}}
		}
		] (C1) -- (B1);
		
	\end{tikzpicture}
\end{center}
ker $\partial_i$ is called the $i-$th cycle, and im $\partial_{i+1}$ is called the $(i+1)-$th boundary. Can see in above diagram, the 2-th boundary (the boundary of triangles) is in 1-th cycle (all 'loops' consists of edges), and this make sense - since any $(i+1)-$th boundary must as well be a cycle.\\
So for simplicial chain complex $\mathcal{C}_{\bullet}$, we have:
\begin{equation*}
	\text{im }\partial_{i+i}\subseteq \text{ker }\partial_{i}
\end{equation*} 
Notices that in general the inclusion is strict, a boundary is itself a cycle, but a cycle need not be a boundary of simplicies! Consider the following example, a 2-D torus (surface of donut) consists of simplcies, We can construct a closed 1-cycle that winds around the hole of the torus, but this cycle is not a boundary of any 2-D surface on the torus!\\
To measure the difference between boundaries and cycles, we define the \textbf{n-th Homology} by the quotient $\mathbb{Z}-$module $\text{ker } \partial_{i}/\text{im }\partial_{i+1}$.

\begin{figure}[h]
	\centering
	\includegraphics[angle=90, width=0.45\textwidth]{cache/Torus_cycles.svg.png}
	\caption{Torus with two closed loops which are not boundaries.}
\end{figure}



\end{document}