\documentclass[english,course]{lecturenotes}
\usepackage{lipsum}
\usepackage{amsmath}
\usepackage{tikz}
\usetikzlibrary{calc}
\usepackage{amsmath,amssymb}
\usepackage{tikz-cd}
\usetikzlibrary{fit}
\usetikzlibrary{arrows.meta}
\usepackage{enumitem}

\title{Algebraic Curves}
\newtheorem*{example*}{Example}
\newtheorem*{proposition*}{Proposition}
\newtheorem*{definition*}{Definition}
\newtheorem*{lemma*}{Lemma}
\newtheorem*{theorem*}{Theorem}
\newtheorem*{question}{Question}

\shorttitle{Shortened title} % For headers; if undefined, the usual title will be used
% Most of these data are not compulsory
\subject{Subject of the Talk}
\author{TSE-YU SU}


\speaker{}
\date{21}{11}{2025}


\begin{document}
	\section{AFFINE ALGEBRAIC SETS}
	\subsection{Affine Space and Algebraic Sets}
	\begin{definition}
		Let $k$ be a field, we define the \textbf{affine n-space} $\mathbb{A}^n(k)$,or  $\mathbb{A}^n$ if $k$ is clear, to be the set of n-tuples of elements of $k$.
	\end{definition}	
	If $F\in k[X_1,...,X_n]$, a point $P=(a_1,...,a_n)$ is said to be a \textbf{zero} of $F$ if\\ $F(P)=F(a_1,...,a_n)=0$. For $F\neq const$, the set of zeros of $F$ is called the \textbf{hypersurface} defined by $F$, and is denoted by $V(F)$. If $F$ is of degree 1, $V(F)$ is called a \textbf{hyperplane} in $\mathbb{A}^n$.\\
	For $S$: a set of polynomials in $k[X_1,...,X_n]$, define $V(S)$ to be the set of all common zeros of $F\in S$.\\
	A subset $X\subseteq \mathbb{A}^n(k)$ is called an \textbf{affine algebraic set}, or simply \textbf{algebraic set}, if $X=V(S)$ for some $S$.\\
	Can verify that the set of all algebraice sets in $\mathbb{A}^n$ forms a topology, called \textbf{Zariski topology}, where the closed sets are exactly the algebraic sets.\\
	Here are some facts about algebraic sets:
	\begin{itemize}
		\item If $I$ is the ideal in $k[X_1,...,X_n]$ generated by $S$, then $V(S)=V(I)$. Therefore we can restrict to the case $V(I)$.
		\item If $\lbrace I_\alpha \rbrace$ is \textbf{any} collection of ideals, then $V(\cup_\alpha I_\alpha)=\cap_\alpha V(I_\alpha)$. So the intersection of \textbf{any} collection of algebraic sets is also an algebraic set.
		\item If $I\subseteq J$, then $V(I)\supseteq V(J)$.
		\item For polynomials $F,G$, $V(F)\cup V(G)=V(FG)$, and for ideals $I,J$, $V(I)\cup V(J)=V(IJ)$. So any \textbf{finite union} of algebraic sets is also algebraic.
		\item $V(0)=\mathbb{A}^n,V(1)=V(k[X_1,...,X_n])=\emptyset$. And $V(X_1-a_1,...,X_n-a_n)=\lbrace  (a_1,...,a_n)\rbrace$. So any finite subset of $\mathbb{A}^n$ is algebraic.
		\item $V(IJ)=V(I\cap J)$. The $\supseteq$ is clear, now prove $\subseteq$. Let $P\in V(IJ)$, then for every $F\in I,G\in J$, $FG(P)=0$. Hence for $H\in I\cap J$, $H^2(P)=0$, so $H(P)=0$. 
	\end{itemize}
	\subsection{The Ideal of a Set of Points}
	For any $X\subseteq \mathbb{A}^n$, consider those polynomials in $k[X_1,...,X_n]$ which vanish on $X$, these polynomials form an ideal $\triangleleft k[X_1,...,X_n]$, called the ideal of $X$, written $I(X)=\lbrace F\in k[X_1,...,X_n] | F(X)=\lbrace 0\rbrace  \rbrace$.
	\\The following properties show some relation between ideals and algebraic sets.
	\begin{itemize}
		\item If $X\subseteq Y$, then $I(X) \supseteq I(Y)$
		\item $I(\emptyset)=k[X_1,...,X_n]$. $I(\mathbb{A}^n)=0$. $I(\mathbb{A}^n)=0$ if $k$ is not finite. A counterexample is $0\neq x(x-1)\in I(\mathbb{A}^1(\mathbb{Z}_2))\triangleleft \mathbb{Z}_2[x]$
		\item $I(\lbrace a_1,...,a_n \rbrace)=(X_1-a_1,...,X_n-a_n)$
		\item $S\subseteq I(V(S))$ and $X\subseteq V(I(X))$
		\item $V(I(V(S)))=V(S)$ and $I(V(I(X)))=I(X)$.
	\end{itemize}
	An ideal which is the ideal of an algebraic set, satisfies the following property:
	\begin{equation*}
		\text{If }I=I(X), \text{ and }F^n\in I \text{ for some }n\in\mathbb{N}, \text{ then }F\in I.
	\end{equation*}
	Consequently, $I(X)=\text{Rad}(I(X))=\sqrt{I(X)}$ is radical. 
	\subsection{The Hilbert Basis Theorem}
	We defined an algebraic set by any set of polynomials, but in fact finitely many will suffice.
	\begin{theorem}
		Every algebraic set is the intersection of a finite number of hypersurfaces.
	\end{theorem}
	In order to prove this theorem, it suffices to show any ideal $I\triangleleft k[X_1,...,X_n]$ is finitely generated by $(F_1,...,F_s)$, then $V(I)=V(F_1)\cap V(F_2)\cap ... \cap V(F_n)$.
	\begin{theorem}[Hilbert Basis Theorem]
		If $R$ is a Noetherian ring, then $R[X]$ is also Noetherian. Consequetly, $R[X_1,...,X_n]$ is Noetherian.
	\end{theorem}
	\subsection{Irreducible Components of an Algebraic Set}
	An algebraic set may be the union of serveral smaller algebraic sets. An algebraic set $V\subseteq \mathbb{A}^n$ is said to be \textbf{reducible} if $V=V_1\cup V_2$, where $V_1\neq V \neq V_2$ are algebraic. Otherwise $V$ is \textbf{irreducible}.
	\begin{proposition}
		An algebraic set $V$ is irreducible if and only if $I(V)$ is prime.
	\end{proposition}
	We want to show that an algebraic set is the union of finitely many irreducible algebraic sets. If $V$ is reducible, write $V=V_1\cup V_2$, if $V_2$ is reducible, write $V_2=V_3\cup V_4$, need to show this process stops.\\
	Since $k[X_1,...,X_n]$ is Noetherian, each set of ideals has an maximal element, consequently, any collection of algebraic sets in $\mathbb{A}^n$ has an minimal element.
	\begin{theorem}
		Let $V$ be an algebraic set in $\mathbb{A}^n(k)$, then there are unique irreducible algebraic sets $V_1,...,V_m$ such that $V=V_1\cup V_2 \cup ... \cup V_m$ and $V_i\nsubseteq V_j$ for $i\neq j$.
	\end{theorem}
	\begin{proof}
	Let $\mathcal{S}$ be the set of all algebraic sets $V\subseteq \mathbb{A}^n$ which is not the union of a finite number of irreducible. Choose an minimal element $V$ in $\mathcal{S}$, clearly $V$ is reducible, say $V=V_1\sup V_2$, where $V_i\neq V$. But then $V_i\subsetneq V$, so $V_i$ is a union of finitely many irreducible algebraic sets, hence so is $V=V_1\cup V_2$, a contradiction. To show $V_i\nsubseteq V_j$ for $i\neq j$, simply delete every algebraic set which is contained in another bigger algebraic set.\\
	To show uniqueness, let $V=W_1\cup ... \cup W_s$. Since $V_i=V\cap V_i=\cup_j (W_j\cap V_i)$, and $V_i$ is irreducible, $V_i\subseteq W_{j(i)}$ for some $j(i)$. Similarly, $W_{j(i)}\subseteq V_{k(j(i))}$ for some $k(j(i))$, but then $V_i\subseteq V_{k(j(i))}$, hence $i=k(j(i))$ and $V_i=W_{j(i)}$. Likewise $W_j=V_{i(j)}$ for some $i(j)$, so $s=n$ and $W_i=V_i$ after renumbering.
	\end{proof}
	These $V_1,...,V_n$ are called the irreducible components of $V$, and $V=V_1\cup ... \cup V_n$ is the decomposition of $V$ into irreducible components.
	\subsection{Algebraic Subsets of the Plane}
	Will classify all irreducible algebraic sets of $\mathbb{A}^2(k)$ in this subsection. Once this classification has been done, by Theorem 1.2 we have found all algebraic sets.\\
	\begin{proposition}
		Let $F,G\in k[X,Y]$ with no common factors. Then $V(F,G)=V(F)\cap V(G)$ is a finite set of points.
	\end{proposition}
	\begin{proof}
		Consider $A=(F,G)\cap k[X]$, can see $A$ is an ideal of $k[X]$. Since $k[X]$ is PID, $A=(f(X))$. So $FH+GK=f(X)$ for some $H,K\in k[X,Y]$. Thus the $X-$component of points in $V(F,G)$ are roots of $f(X)$, which is finitely many. Similarly, the $Y-$component of points in $V(F,G)$ are roots of some $g\in k[Y]$. Hence $V(F,G)\subseteq \lbrace (a,b) | f(a)=g(b)=0 \rbrace$, which is finite.
	\end{proof}
	\begin{corollary}
		If $F$ is irreducible in $k[X,Y]$, and if $V(F)$ is infinite, then $I(V(F))=(F)$ and $V(F)$ is irreducible.
	\end{corollary}
	\begin{proof}
		Take $G\in I(V(F))$, clearly $G(V(F))=\lbrace 0 \rbrace$, hence $V(F)\subseteq V(F,G)$, and $V(F,G)$ is infinite. By the previous proposition $F,G$ must have common factor, since $F$ is irreducible this common factor can only be $F$, so $G\in (F)$, and thus $I(V(F))=(F)$, and by proposition 1.4 $V(F)$ is irreducible.
	\end{proof}
	\begin{corollary}
		Suppose $k$ is infinite, then the irreducible algebraic subsets of $\mathbb{A}^2(k)$ are:
		\begin{align*}
			&\mathbb{A}^2(k),\\
			&\emptyset,\\
			&\textbf{points},\\
			&\textbf{irreducible plane curves }V(F)
		\end{align*}
	where $F$ is an irreducible polynomial and $V(F)$ is infinite.
	\end{corollary}
	\textbf{Note:} Not all zero sets of irreducible polynomial in $k[X,Y]$ is infinte, for example $X^2+Y^2\in\mathbb{R}[X,Y]$ is irreducible, but the zero set $\lbrace (0,0) \rbrace$ is finite.
	\begin{corollary}
		Assume $k$ is algebraically closed, and $F\in k[X,Y]$. Let $F=F_{1}^{n_1}...F_{r}^{n_r}$ be the decomposition of $F$ into irreducible factors. Then $V(F)=V(F_1)\cup ... \cup V(F_r)$ is the decomposition of $F$ into irreducible components, and $I(V(F))=(F_1F_2...F_r)$.
	\end{corollary}
	\begin{proof}
		$V(F)=V(F_1)\cup ... \cup V(F_r)$ is clear. Since $k$ is algebraically closed, $V(F_i)$ is infinite, and by the previous corollary $V(F_i)$ is irreducible.\\
		(Note: The cases such as $X^2+Y^2\in \mathbb{R}[X,Y]$, which is irreducible but has finite zero set, won't happen.)\\ 
		Also, since $F_i \nmid F_j$, there's no inclusion relation among $V(F_i)$.\\
		The next part $I(V(F))=(F_1F_2...F_r)$ is also clear.
		
	\end{proof}
	The following problem shows why we usually require $k$ to be algebraically closes.
	\begin{question}
		Show that every algebraic subset of $\mathbb{A}^2(\mathbb{R})$ is equal to some $V(F)$, where $F\in\mathbb{R}[X,Y]$.
	\end{question}
	\begin{proof}
		It suufices to show any finite set of points $\lbrace (a_1,b_1),...,(a_r,b_r) \rbrace$ in $\mathbb{A}^2(\mathbb{R})$ can be written as $V(F)$ for some $F\in\mathbb{R}[X,Y]$.\\
		Since $(X-a)^2+(Y-b)^2$ has only one zero $(a,b)$ in $\mathbb{A}^2(\mathbb{R})$, $F=\prod_{i=i}^{r}((X-a_i)^2+(Y-b_i)^2)$ is the desired polynomial.
	\end{proof}
	\subsection{Hilbert's Nullstellensatz}
		\begin{center}\fbox{we assume $k$ is algebraically closed in this subsection}.\end{center}\\
		Want to find the exact relation between algebraic sets and ideals. Will first prove a weaker theorem: 
		
	\begin{theorem}[Weak Nullstellensatz]
		If $I$ is a proper ideal in $k[X_1,...,X_n]$, then $V(I)\neq \emptyset$.
	\end{theorem}
	\begin{proof}
		Since $I$ is contained in some maximal ideal $\mathfrak{m}$, and $V(\mathfrak{m})\subseteq V(I)$, it suffices to show for every maximal ideals $\mathfrak{m}$, $V(\mathfrak{m})\neq \emptyset$.\\
		Will use the following fact:\\
		\textbf{Fact:} If $k$ is algebraically closed, then maximal ideals of $k[X_1,...,X_n]$ are of the form $(X_1-a_1,...,X_n-a_n)$.\\
		By the above fact $V(X_1-a_1,...,X_n-a_n)=\lbrace (a_1,...,a_n) \rbrace\neq \emptyset$.
	\end{proof}
	\begin{theorem}[Hilbert's Nullstellensatz]
		Let $I$ be an ideal in $k[X_1,...,X_n]$, $k$ is algebraically closed. Then $I(V(I))=\text{Rad}(I)$.
	\end{theorem}
	\begin{proof}
		Rad$(I)\subseteq I(V(I))$ is easy. For another direction, suppose $G\in I(V(F_1,...,F_r)),$ $F_i\in k[X_1,...,X_n]$, let $J=(F_1,...,F_r,X_{n+1}G-1)\subseteq k[X_1,...,X_n,X_{n+1}]$, can see $V(J)\subseteq \mathbb{A}^n=\emptyset.$ Apply Weak Nullstellensatz to $J$, $J=k[X_1,...,X_n,X_{n+1}]$. So $1=\sum A_i(X_1,...,X_{n+1})F_i+B(X_1,...,X_{n+1})\cdot(X_{n+1}G-1)$.\\
		Let $Y=\frac{1}{X_{n+1}}$, multiply the above equation sufficiently many times by $Y$, that the $X_{n+1}$-degree of each monomial terms is negative. (For example, $X_1 X_{n+1}^{3}+X_{2}^{3} X_{n+1}^{5}\xrightarrow{\times Y^5}X_1Y^2+X_{2}^{3}=P(\lbrace X_i|i=1,...,n \rbrace,Y)$ )\\
		Then we get an equation $Y^N=\sum C_i(X_1,...,Y)F_i+D(X_1,...,X_n,Y)\cdot(G-Y)\in k[X_1,...,X_n,Y]$, substitute $Y=G$, it follows that $G^N\in (F_1,...,F_r)$.
	\end{proof}
	Here are some immediate corollary, for $k:$ algebraically closed:
	\begin{corollary}
		There is a one-to-one correspondence between \textbf{radical ideals} and \textbf{algebraic sets}.
	\end{corollary}
	\begin{corollary}
		If $I$ is prime, then $V(I)$ is irreducible. There is a one-to-one correspondence between \textbf{prime ideals} and \textbf{irreducible algebraic sets}. The maximal ideals correspond to points.
	\end{corollary}
	\begin{corollary}
		Let $F=F_{1}^{n_1}...F_{r}^{n_r}$ be the decomposition of $F$ into irreducible factors, then $V(F)=V(F_1)\cup ... \cup V(F_r)$ is the decomposition of $V(F)$ into irreducible components, and $I(V(F))=(F_1F_2...F_r)$. \\
		There is a one-to-one correspondence between \textbf{irreducible polynomials}(up to multiplying by a unit) and \textbf{irreducible hypersurfaces} in $\mathbb{A}^n(k)$. Remenber that a hypersurface is the zero set of a polynomial.
	\end{corollary}
	\begin{align*}
		\text{Radical ideals }&\leftrightarrow \text{ Algebraic sets}\\
		\text{Prime ideals }&\leftrightarrow \text{ Irreducible algebraic sets}\\
		\text{Irreducible polynomials }&\leftrightarrow \text{ Irreducible hypersurfaces}
	\end{align*}
	\begin{corollary}
		Let $I$ be an ideal in $k[X_1,...,X_n]$, then $V(I)$ is a finite set if and only if $k[X_1,...,X_n]/I$ is a finite dimensional vector space over $k$. In this case the number of points in $V(I)$ is less or equal to $ \text{dim}_k(k[X_1,...,X_n]/I)$.
	\end{corollary}
	\begin{proof}
		Assume $k[X_1,...,X_n]/I$ is a finite dimensional vector space over $k$, let points $P_1,...,P_r\in V(I)$, choose polynomials $F_i\in k[X_1,...,X_n],i=1,...,r$ s.t. $F_i(P_i)=1$ and $F_i(P_j)=0$ for $i\neq j$. Want to show $I-$redidue classes $\bar{F_i}$ are linearly independent over $k$. If $\sum_{i=1}^{r}\lambda_i \bar{F_i}=0$, then $\sum_{i=1}^{r}\lambda_i F_i\in I$, so $\lambda_j=\sum_{i=1}^{r}\lambda_i F_i(P_j)=0$, hence $\bar{F_i}$ are linearly independent over $k$, so $r\leq \text{dim}_k(k[X_1,...,X_n]/I)$.\\
		Conversely, if $V(I)=\lbrace P_1,...,P_r \rbrace$ is finite, let $P_{i}=(a_{i,1},...a_{i,n})$, and for $j=1,...,n$ define $F_j=\prod_{s=1}^{r}(X_j-a_{s,j})$, clearly $F_j\in V(I)$, so by Nullstellensatz $F_j^N\in I$ for some $N$, WLOG take $N$ so large that it holds for all $F_j$. Consequently $\bar{F_{j}^{N}}=0$, for all $j$ and
		 since $F_{j}^{N}\in I$ is a polynomial in $X_j$ of degree $rN$, $\bar{X_j}^{rN}$ is a $k-$linear combination of $1,X_{j}^{1},X_{j}^{2},...,X_{j}^{rN-1}$, and hence so is any positve order $X_{j}^{s}$. Therefore $\lbrace \bar{X_1}^{m_1}\cdot \bar{X_2}^{m_2}\cdot...\cdot \bar{X_n}^{m_n} \rbrace$ generates $k[X_1,...,X_n]/I$ as a vector space over $k$.
	\end{proof}
	\subsection{Gröbner basis}
	For polynomials with one variable $f(x),g(X)\in k[X]$, we have polynomial division algorithm $f(X)=g(X)q(X)+r(X)$, where the degree of the remainder $r(X)$ is less than $g(X)$. But for multivariate polynomials, there's no natural ordering of polynomials. So in order to do \textbf{generalized polynomial division}, we need an \textbf{monomial ordering}, usually the lexico order $x_1>x_2>...$ or some permutation $x_{\alpha_1}>x_{\alpha_2}>...$.\\
	We also need another tool called Gröbner basis to perform generalized polynomial division, Gröbner basis is strongly related to some well-known algorithm, such as:
	\begin{itemize}
		\item Gaussian Elimination
		\item Euclidean Algorithm for Computing gcd
		\item Simplex Algorithm
		\item ... 
	\end{itemize}
	\begin{center} 
	\begin{tikzpicture}
		
		\node (X1) at (4,0) {\fbox{Set of input polynomials $F=\lbrace f_1,...,f_n \rbrace$}};
		\node (X2) at (4,-2) {\fbox{Set of output polynomials $G=\lbrace g_1,...,g_m \rbrace$(Gröbner basis)}};

		
		\draw[->] (X1)--(X2) node[midway,right]{Buchberger Algorithm};
	\end{tikzpicture}
	\end{center}
	\begin{center}
			
			\begin{tikzpicture}
			
			\node (X1) at (4,0) {$\lbrace 2x+3y+4z-5,3x+4y+5z-2 \rbrace$:Two plane in $\mathbb{R}^3$};
			\node (X2) at (4,-2) {$\lbrace x-z+14,y+2z-11  \rbrace$:Two line in $x-y,y-z$ plane, respectively};
			
			
			\draw[->] (X1)--(X2) node[midway,right]{Buchberger Algorithm};
		\end{tikzpicture}
	\end{center}
	Now give the definition of Gröbner basis, first fix a lexico ordering $X_1>X_2>...$.
	\begin{definition}[Initial Monomial,Ideal]
		Let $>$ be a monomial ordering, define the \textbf{initial monomial} $in_{>}(F)$ to be the leading term of the polynomial $F\in k[X_1,...,X_n]$ i.e. the monomial in $F$ of maximal degree.\\
		Consider the ideal consists of every initial monomial of polynomial in $I$, called the \textbf{initial ideal}, denoted by $in_>(I)$. Can see $in_>(I)$ is indeed an ideal in $k[X_1,...,X_n]$.
	\end{definition}
	\begin{definition}[Gröbner basis]
		Given polynomials $f_1,...,f_n$ and monomial ordering $>$, a finite set of polynomial $\lbrace g_1,g_2,...,g_m \rbrace$ is called a \textbf{Gröbner basis} of $\lbrace f_1,...,f_n \rbrace$ if $in_>(f_1,...,f_n)=(in_>(g_1),...,in_>(g_m))$.
	\end{definition}
	Will show an combinatory application of Gröbner basis:
	\begin{question}
		Minimize linear function $P+N+D+Q$, while $P,N,D,Q\in\mathbb{N}\cup \lbrace 0 \rbrace$, with constraint $ P+5N+10D+25Q=117$.
	\end{question}
	\textbf{Answer:} Consider polynomial ring $\mathbb{Q}[C,P,N,D,Q]$, can represent a combination of coins by a monomial $C^{n}P^{n_P}N^{n_N}D^{n_D}Q^{n_Q}$, where $n=\sum_\alpha n_\alpha$.\\
	Now define relations:
	\begin{align*}
		&f_1=CN-C^5P^5\\
		&f_2=CD-C^{10}P^{10}\\
		&f_3=CQ-C^{25}P^{25}
	\end{align*}
	These three polynomial represent all the exchange rules of coins, which are:
	\begin{align*}
		&\text{1 nickel }=\text{ 5 pennies}\\
		&\text{1 dime }=\text{ 10 pennies}\\
		&\text{1 quarter }=\text{ 25 pennies}\\
	\end{align*}
	The $C-$degree of each monomial in the relations represents the number of coins.\\
	Now choose an ordering $C> P> N> D> Q$, then apply Buchberger Algorithm to these three polynomials, will get a unique \textbf{reduced Gröbner basis} consists of nine polynomials (computed by program, might be wrong):
	\begin{align*}
		&g_1: C^5P^5 - CN\\
		&g_2: C^2N^2 - CD\\
		&g_3: C^3ND^2 - CQ\\
		&g_4: C^4P^5D - CN^3\\
		&g_5: C^2D^3 - CNQ\\
		&g_6: CN^3Q - CD^4\\
		&g_7: C^3P^5D^2 - CN^5\\
		&g_8: CP^5Q - CN^6\\
		&g_9: CP^5D^4 - CN^9\\
	\end{align*}
	In order to minimize total number of coins, we apply generalized polynomial division to $C^{117}P^{117}$(represents 117 coins of penny), and calculate the remainder, which is:
	\begin{center}
		\fbox{$C^8P^2NDQ^4$}
	\end{center}
	The multi-degree $(8,2,1,1,4)$ means total of 8 coins, with 2 pennies, 1 nickel, 1 dime, and 4 quarters.\\
	Here are some total amount of cents and its minimal number of coins represented with monomials:\\
	\begin{center}
\begin{tabular}{|c|c|}
	\hline
	Dividend & Remainder \\
	\hline
	$C^{30}P^{30}$ & $C^{2}NQ$ \\
	$C^{567}P^{567}$ & $C^{26}P^{2}NDQ^{22}$ \\
	$C^{9999}P^{9999}$ & $C^{405}P^{4}D^{2}Q^{399}$ \\
	$C^{35857}P^{35857}$ & $C^{1437}P^{2}NQ^{1434}$ \\
	$C^{8679031}P^{8679031}$ & $C^{347163}PNQ^{347161}$ \\
	\hline
\end{tabular}
\end{center}
\textbf{Note:} The additional variable $C$ for counting number of coins is crucial, one may try to work under $\mathbb{Q}[P,N,D,Q]$ and compute Gröbner basis for:
\begin{align*}
	&f^{\prime}_{1}=N-P^5\\
	&f^{\prime}_{2}=D-P^{10}\\
	&f^{\prime}_{3}=Q-P^{25}
\end{align*}
with ordering $P>N>D>Q$, this is the only reasonable choice of lexico order, since generalized polynomial division tend to make the remainder as small as possible, and the least favorite coin is penny, otherwise the remainder of, for example $Q$ may be $P^{25}$, in which case the total number of coins increases. Similarly for $N,D,Q$. But one observes that:
\begin{equation*}
	QN\sim D^3 
\end{equation*}
Which the degree on the left-hand-side is greater, but the right one has much number of coins.
\\
	

\end{document}