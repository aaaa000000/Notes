\documentclass[12pt]{article}

\usepackage{amsmath,amssymb,amsthm}
\usepackage[a4paper,margin=2.5cm]{geometry}
\usepackage{amsmath}
\usepackage{mathtools}


\begin{document}
	
\section*{Solution}
	Let
	\[
	(1+\sqrt{2})^n=a_n+b_n\sqrt{2}\in \mathbb{Z}(\sqrt{2})
	\]
	Since
	\[
	 (\sqrt{2}+1)^{2026}+(\sqrt{2}-1)^{2026}=2a_n,\text{ and } \sqrt{2}-1<1
	\]
	Hence
	\[
	\lfloor (1+\sqrt{2})^{2026} \rfloor=\lfloor 2a_n-(\sqrt{2}-1)^{2026} \rfloor=2a_n-1
	\]
	There is a recursion relation:
	\[
	\begin{pmatrix}
		a_{n+1} \\[4pt]
		b_{n+1}
	\end{pmatrix}
	=
	\begin{pmatrix}
		1 & 2 \\
		1 & 1
	\end{pmatrix}
	\begin{pmatrix}
		a_n \\[4pt]
		b_n
	\end{pmatrix}
	\]
	Consider the vecotr $(a_n,b_n)^T$ mod 10:
	\begin{equation*}
		\begin{aligned}
			(1,0)^T 
			&\rightarrow (1,1)^T \\
			&\rightarrow (3,2)^T \\
			&\rightarrow (7,5)^T \\
			&\rightarrow (17,12)^T\equiv(7,2)\\
			&\rightarrow (11,9)\equiv(1,9)\\
			&\rightarrow (19,10)\equiv(-1,0)
		\end{aligned}
	\end{equation*}
	So the cycle is of length $12$. Let $\tilde{a}_n\coloneqq a \text{ (mod 10)}$. \\
	Sincd $2026=12\cdot 168 +10$, $a_{2026}=\tilde{a}_{10}\equiv -7 \equiv 3$, the answer is $2\cdot 3 -1=5$.
	
	
	
\end{document}
